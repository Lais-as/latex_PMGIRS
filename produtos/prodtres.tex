	\thispagestyle{headfootimage}
	\section{Introdução - Produto 3	}
	
	Este capítulo consiste no levantamento e análise do manejo dos resíduos sólidos gerados no município de Monteiro Lobato considerando a caracterização dos resíduos segundo a origem, o volume, as formas de destinação, a disposição final adotada e os respectivos custos associados. Além disso, descreve o acondicionamento, a coleta, o transporte e o transbordo, este quando aplicável, dos seguintes tipos de resíduos:
	
	\begin{itemize}
		\item [primeiro item] Resíduos Domiciliares;
		\item Resíduos de Limpeza Urbana;
		\item Resíduos de Estabelecimentos Comerciais e Prestadores de Serviços;
		\item Resíduos dos Serviços Públicos de Saneamento;
		\item Resíduos Industriais;
		\item Resíduos de Serviços de Saúde;
		\item Resíduos da Construção Civil;
		\item Resíduos Agrossilvipastoris;
		\item Resíduos de Serviços de Transportes; 
		\item Resíduos de Mineração; e
		\item Resíduos de Logística Reversa.
	\end{itemize}
	
	Para a elaboração deste diagnóstico, foi realizada uma reunião preliminar com a \acrlong{smaa} (\acrshort{smaa}), \acrlong{ssm} (\acrshort{ssm}) e vice-prefeito para apresentação do objetivo da elaboração do \gls{pmgirs} de Monteiro Lobato e para compreender as responsabilidades e competências de cada secretaria. Após a reunião, foram encaminhados questionário dirigidos em forma de ofícios para as secretarias com questionamentos detalhados sobre a forma de gestão e gerenciamento dos resíduos sólidos gerados no município. As secretarias realizaram reuniões conjuntas para responder os ofícios e retornaram as respostas por meio eletrônico.
	
	A fim de complementar as informações secundárias apresentadas, foram realizadas visitas de campo acompanhadas pela \gls{smaa} para registros fotográficos do gerenciamento dos resíduos, realização de questionários estruturados para ampliar a participação dos munícipes na elaboração do \gls{pmgirs} e acompanhamento dos serviços de coleta dos resíduos comum e reciclável para determinação da rota.
	
	Os trajetos da coleta e destinação final dos resíduos foram determinados por coordenadas geográficas através do \gls{gps}, modelo Garmin etrex. Esses pontos foram inseridos no software ArcGIS e sobrepostos ao mapa da região.
	
	O questionário estruturado foi aplicado aos estabelecimentos comerciais responsáveis pela geração de óleos lubrificantes e pneus cujos endereços foram disponibilizados pela Prefeitura. O questionário foi realizado para conhecimento da geração e destinação final destes resíduos que devem apresentar sistema de logística reversa de acordo com o artigo 33 da Lei nº 12.305, de 02 de agosto de 2010.
	
	O município de Monteiro Lobato não dispõe de recurso específico destinado para o gerenciamento dos resíduos sólidos e de limpeza urbana. Neste contexto, foi solicitado à \acrshort{ssm}, via ofício, os custos associados para todas atividades do manejo de resíduos sólidos e serviços de limpeza urbana em relação aos salários dos funcionários, manutenção de equipamentos, contratos com empresas terceirizadas e gastos com transporte.
	
	A \acrshort{ssm} repassou um documento em formato .pdf com a relação de funções e atividades de vinte e nove funcionário e a folha de ponto. A fim de estimar qual o custo dos funcionários para a prefeitura em cada atividade de manejo de resíduos sólidos e de serviços de limpeza urbana, o valor total do salário base foi somado aos descontos pagos pela prefeitura (imposto de renda, \acrshort{inss} e demais contribuições) e ao \acrshort{fgts} e este total foi dividido entre os vinte e nove funcionários.
	
	Dependendo do cargo, das atribuições e das horas extras realizadas, a prefeitura ainda arca com uma base variável mensal em relação aos funcionários referente ao pagamento de horas extras e Salário-família. De acordo com a descrição da folha de pagamento disponibilizada, esta base variável é de aproximadamente R\$ 7.202,37 ao mês.
	
	Além disso, o município dispõe de cinco funcionários do Programa Frente de Trabalho, que qualifica profissionalmente e gera renda para cidadãos que se encontram desempregados e em situação de alta vulnerabilidade social, para colaboração em serviços de limpeza urbana, serviços de manutenção e serviços braçais. A estimativa do valor destinado a estes funcionários foi realizada considerando o salário base e o valor da cesta básica que o município disponibiliza para os funcionários.
	
	A partir do diagnóstico de resíduos sólidos é possível desenvolver um sistema de cálculo específico de custos dos serviços de limpeza urbana e de manejo de resíduos sólidos. A partir disso e da elaboração efetiva do \gls{pmgirs} de Monteiro Lobato, o município tem direito a ter acesso a recursos da União destinados para serviços relacionados a limpeza urbana e ao manejo de resíduos sólidos, de acordo com o artigo 18 da \gls{pnrs} (BRASIL, 2010a).
	
	O diagnóstico da situação de resíduos no município de Monteiro Lobato é indispensável para a definição de futuras diretrizes, identificação de gerenciamento inadequados, para o levantamento de ações mitigadoras e preventivas e para a elaboração de prognóstico (BRASIL, 2007a). Além disso, o levantamento do diagnóstico é de suma importância para a elaboração de propostas que visem a utilização racional de recursos ambientais, a redução de desperdícios, a minimização da geração de resíduos sólidos e o manejo ambientalmente adequado e economicamente viável.
	
	\section{Diagnóstico do sistema de limpeza urbana e manejo de resíduos sólidos}
	
	De forma resumida, o artigo 13 da Lei nº 12.305, de 02 de agosto de 2010 classifica os resíduos sólidos como:
	
	\begin{description}
		\item[Resíduos Domiciliares] originários de atividades domésticas em residências urbanas;
		\item[Resíduos de Limpeza Urbana] originários da varrição, limpeza de logradouros e vias públicas e outros serviços de limpeza urbana;
		\item[Resíduos Sólidos Urbanos] englobados nos resíduos domiciliares e nos resíduos de limpeza urbana;
		\item[Resíduos de Estabelecimentos Comerciais e Prestadores de Serviços] gerados nessas atividades, exceto os serviços de limpeza urbana, de saneamento básico, de saúde, de construção civil e de transportes;
		\item[Resíduos de Serviços Públicos de Saneamento Básico] gerados nessas atividades, exceto os resíduos sólidos urbanos;
		\item[Resíduos Industriais] gerados nos processos produtivos e instalações industriais;
		\item[Resíduos de Serviços de Saúde] gerados nos serviços de saúde, conforme definido em regulamento ou em normas estabelecidas pelos órgãos do Sisnama (Sistema Nacional do Meio Ambiente) e do SNVS (Sistema Nacional de Vigilância Sanitária);
		\item[Resíduos da Construção Civil] gerados nas construções, reformas, reparos e demolições de obras de construção civil, incluídos os resultantes da preparação e escavação de terrenos para obras civis;
		\item[Resíduos Agrossilvipastoris] gerados nas atividades agropecuárias e silviculturais, incluídos os relacionados a insumos utilizados nessas atividades;
		\item[Resíduos de Serviços de Transportes] originários de portos, aeroportos, terminais alfandegários, rodoviários e ferroviários e passagens de fronteira;
		\item[Resíduos de Mineração] gerados na atividade de pesquisa, extração ou beneficiamento de minérios;
		\item[Resíduos de Logística Reversa] gerados após o uso pelo consumidor dos materiais provindos de fabricantes, importadores, distribuidores e comerciantes de agrotóxicos e suas embalagens, pilhas e baterias, óleos lubrificantes e suas embalagens, lâmpadas fluorescentes de vapor de sódio e mercúrio e de luz mista, produtos eletroeletrônicos e seus componentes.
		
	\end{description}
	
	\subsection{Resíduos Sólidos Urbanos}
	Os Resíduos Sólidos Urbanos (RSU) compreendem os resíduos domiciliares e os resíduos de limpeza urbana. Os resíduos domiciliares são os resíduos originários de atividades domésticas em residências urbanas. Os resíduos de limpeza urbana englobam os resíduos originários de varrição, de limpeza de logradouros, de vias públicas e de capina e poda (BRASIL, 2010a; BRASIL, 2007a).
	
	Além disso, nos casos em que os resíduos de estabelecimentos comerciais e prestadores de serviços são caracterizados como não perigosos, o poder público pode classificá-los como resíduos equivalentes aos resíduos domiciliares (BRASIL, 2010a).
	
	A geração estimada pela \gls{abrelpe} de RSU no Brasil foi de 78,3 milhões de toneladas em 2016, considerando a soma das projeções de cada região do país. Já a geração per capita apresentou um valor de aproximadamente 1,04 kg por habitante por dia (ABRELPE, 2016).
	
	Em relação a cobertura de coleta, o \gls{ipea} publicou em 2012 uma relação da coleta direta e indireta em área urbana e rural de resíduos sólidos urbanos entre os anos de 2000 e 2008 (tabela 1).
	
	Analisando a tabela 1 é possível verificar que a distribuição da taxa de cobertura de coleta de RSU não é uniforme entre as regiões do país. Apesar de a coleta em área rural ter duplicado de 2000 a 2008 em porcentagem, a cobertura é muito baixa, sendo apenas 32,7 \% no país (IPEA, 2012a).
	
	As regiões Sudeste e Sul apresentaram a maior cobertura em área rural no ano de 2008, porém esses valores representam apenas 50 \% dos domicílios. Já a cobertura na área urbana apresenta valores acima de 95 \% de abrangência para todas as regiões do país. No entanto, o índice de coleta total das regiões Norte, Nordeste e Centro-Oeste ainda possuem deficiência (IPEA, 2012a).
	
	Em 2016, de acordo com as estimativas da ABRELPE, o índice de cobertura de coleta de RSU do país cresceu para 91 \% e os índices da região Norte, Nordeste, Centro-Oeste, Sudeste e Sul aumentaram para, respectivamente, 81 \%, 79 \%, 94 \%, 98 \% e 95 \% (ABRELPE, 2016). Comparando as estimativas de 2008 da tabela 1 com as de 2016, a abrangência de coleta de RSU foi ampliada, porém, a distribuição ainda não é equivalente entre todas as regiões.
	Tratando-se de disposição final no Brasil referente aos resíduos sólidos domiciliares e/ou urbanos, a sua relação e diferentes formas são apresentadas na figura 1, com dados de tonelada por dia para o ano de 2008.A disposição final mais utilizada no Brasil, assim como pode ser observado na figura 1, ainda se trata dos aterros sanitários. 
	
	Estima-se que, em 2016, 58,4 \% de RSU do montante total do país tenha sido destinado a aterros sanitários, enquanto que 24,2 \% foi disposto em aterro controlado e 17,4 \% em lixões (ABRELPE, 2016). Neste contexto, é possível aferir que não houve significativa evolução em relação a disposição ambientalmente adequada dos resíduos sólidos no país.
	
	Apesar do Brasil ter um grande número de municípios contemplados com a disposição final dos RSU em aterros sanitários, isso representa 40,20 \% do total, ou seja, menos da metade dos municípios brasileiros possuem aterros considerados sanitários. Quando se fragmenta, na Figura S, essa distribuição por regiões, percebe-se que as regiões Norte e Nordeste possuem um maior número de locais com disposição final ambientalmente inadequadas. A região Centro-Oeste apresenta uma relação numericamente comparável entre as formas atualmente adotadas de disposição final. As regiões Sudeste e Sul têm as melhores condições quanto à disposição final dos RSU, e possui lixões em 10 \% e 12 \% dos municípios, respectivamente.
	
	O serviço público é responsável pela gestão e gerenciamento dos serviços de limpeza urbana e pelo manejo dos resíduos domiciliares, de modo a atender ao plano municipal de gestão integrada de resíduos sólidos (BRASIL, 2010a).
	
	Se o poder público classificar os resíduos de estabelecimentos comerciais e prestadores de serviço como equivalentes à resíduos domiciliares, o gerenciamento será de responsabilidade do poder público. No entanto, se os resíduos de estabelecimentos comerciais e prestadores de serviço forem caracterizados como perigosos ou não forem equiparados à resíduos domiciliares por sua natureza, composição, volume ou quantidade superior a determinada pelo município, os geradores estarão sujeitos a elaborar o próprio plano de gerenciamento (BRASIL, 2010a).
	
	No município de Monteiro Lobato os RSU são divididos em duas categorias: resíduos sólidos comuns e resíduos sólidos recicláveis. Os resíduos sólidos comuns no município consistem em resíduos considerados domiciliares (com exceção de recicláveis), rejeitos e matéria orgânica. 
	Os resíduos oriundos de serviços de limpeza urbana como varrição, desobstrução de sarjetas, poda e capina são usualmente considerados pelo município como resíduos comuns e recebem, em sua maioria, a mesma destinação e disposição final que os resíduos domiciliares e de estabelecimentos comerciais e prestadores de serviços.
	
	Os resíduos sólidos recicláveis são constituídos de materiais plásticos de diversas categorias (PET, PP, PEBD, PEAD entre outros), vidro, metal, papel e papelão, possuem valor agregado e devem voltar ao sistema produtivo como matéria prima para fabricação de novos produtos através da reciclagem e/ou reaproveitamento.
	
	\subsection{Resíduo Sólido Comum}
	
	Em relação à cobertura de coleta de RSU do município de Monteiro Lobato, a tabela 2 apresenta a relação entre população (total, urbana e rural) atendida pelo serviço de coleta, dados disponibilizados entre 2009 e 2015 (SNIS, 2009 a 2015).
	
	Ao analisar a tabela 2, verifica-se a deficiência na declaração dos dados pelos municípios para o Sistema Nacional de Informações sobre Saneamento (SNIS). A ausência de dados consistentes ao longo da Série Histórica prejudica a interpretação e expõe uma fragilidade em relação a qualidade das informações.
	
	Considerando-se a fragilidade dos dados de cobertura de coleta, as discussões serão baseadas principalmente nas informações prestadas pela autoridade pública municipal de Monteiro Lobato.
	
	Os dados obtidos sobre os resíduos comuns foram coletados em três etapas. Primeiro foram encaminhados questionários dirigidos em forma de ofício para SMAA bem como para a SSM e separados por origem de resíduos: RSU, Resíduos de Serviço de Saúde (RSS), Resíduos de Construção Civil (RCC). Nesse questionário foi indagado sobre as condições de coleta, equipamentos utilizados, número de funcionários responsáveis pelas atividades, custo associado, destinação e disposição final dos resíduos.
	
	Em seguida, na segunda etapa através da ferramenta Excel 2016, foram realizadas avaliações estatísticas dos dados de disposição final destes resíduos no aterro sanitário de Tremembé administrado pela companhia ESTRE-Resicontrol, que forneceu uma série histórica entre 2 de janeiro de 2014 até 9 de outubro de 2017.
	
	Por fim, foram realizados registros fotográficos das formas de acondicionamento dos resíduos comum e dos equipamentos e materiais utilizados, a fim de validar as respostas dos questionários dirigidos.
	
	\subsubsection{Acondicionamento}
	
	De acordo a SSM, o acondicionamento coletivo de resíduos comuns no município é composto por 59 lixeiras. Os resíduos comuns domiciliares e comerciais ficam normalmente armazenados em lixeira de ferro de, aproximadamente, 2 x 1 x 1 metros, como apresentado na parte superior da figura 2. Existem também as lixeiras de madeira, que apresentam dimensões de 2 x 1 x 1 m ou de 1,20 x 1 x 1 m (parte inferior da figura 2). Foi possível verificar que, apesar de as lixeiras apresentarem um volume significativo, alguns resíduos são dispostos inadequadamente no solo ao lado das lixeiras. 
	
	Os resíduos introduzidos nas lixeiras são embalados pelos munícipes em sacolas de supermercado, saco plástico pretos de volumes distintos, caixas de papelão entre outros materiais. Entretanto, muitos resíduos sólidos são dispostos nas lixeiras sem o devido acondicionamento, o que prejudica o processo de coleta por parte dos funcionários, como mostrado na Figura 3.
	
	As lixeiras maiores costumam compor as áreas comuns dos bairros, no entanto, alguns moradores possuem o próprio cesto de lixo, como exemplificado pela figura 6. Existe ainda o hábito de alguns moradores pendurarem sacos plásticos com os resíduos em ganchos no portão da residência, como apresentado na figura 7.
	
	\subsubsection{Coleta, transbordo e transporte}
	
	Os resíduos comuns coletados no município são compostos por resíduos sólidos produzidos por estabelecimentos comerciais e prestadores de serviço, residências de área urbana e área rural, resíduos de caráter domiciliar não reciclável de estabelecimentos particulares geradores de RSS, Resíduos de Serviços de Transporte (RST) e Resíduos Industriais (RI), além de parte dos resíduos oriundos da limpeza pública. Essa abrangência na coleta ocorre devido à inexistência de regulamentação pelo município quanto ao tipo de gerador e suas obrigações legais conforme preconiza a PNRS (Lei nº 12.305 de 2010).
	
	A coleta municipal dos resíduos sólidos comuns ocorre às segundas, terças, quintas e sextas-feiras, de acordo com os horários apresentados na tabela 3.  O município dispõe de um caminhão do tipo compactador de 19 m$^{3}$ (vide figura 8) com o qual é realizada a coleta.
	
	Das 59 lixeiras coletivas existentes no município e conforme descrição da SSM, o caminhão coletor passa pelos seguintes bairros com respectivos números de lixeiras coletivas após sair da SSM:
	
	\begin{itemize}
		\item SP 50 Bairro Taquari – 4 lixeiras; 
		Bairro Descoberto – 1 lixeira; 
		Bairro Ponte Nova – 3 lixeiras; 
		Estrada do Livro – 2 lixeiras; 
		Bairro Pedra Branca – 5 lixeiras; 
		Estrada São Francisco – 6 lixeiras; 
		Centro – 12 lixeiras; 
		Bairro Alpes do Boquira – 1- lixeira; 
		SP 50 Bairro São Benedito – 13 lixeiras; 
		Estrada Ito Rennó – 2 lixeiras; 
		Estrada Cruzeiro – 1 lixeira; 
		Bairro do Souzas – 4 lixeiras; 
		Estrada Fabiano – 2 lixeiras; 
		Estrada Pandavas – 3 lixeiras.
	\end{itemize}

	Durante o percurso realizado em uma segunda-feira foram obtidas as coordenadas geográficas das lixeiras cujo resíduo sólido comum foi recolhido através do GPS. A rota dessa coleta passou por 37 lixeiras cujas localizações são apresentadas na figura 10. 
	
	A coleta dos pontos geográficos das lixeiras ocorreu juntamente com o registro fotográfico de cada local.  Durante o percurso, notou-se que em alguns pontos há o descarte, em local inadequado, do líquido acumulado no caminhão compactador, conforme ilustra a figura 9.
	
	Para a retirada dos resíduos das lixeiras é feita uma manobra com o caminhão, em que o mesmo se aproxima de ré até ficar bastante próximo da lixeira, como ilustram as figuras 11 e 12. A figura 12 também mostra a disposição inadequada dos resíduos sólidos no solo, fora da lixeira.
	
	Em dezembro de 2017, o percurso do caminhão coletor foi acompanhado durante alguns dias da semana e sua rota registrada através do GPS. A obtenção dessas rotas colabora com o possível traçado de caminhos mais otimizados, e assim, o município teria um transporte mais rápido e com menor consumo de combustível, gerando menor custo e menor emissão de gases de efeito estufa.
	
	Para os dias de coleta de resíduo comum (segunda, terça, quinta e sexta-feira), não há uma rota pré-definida pela secretaria, ficando a critério do motorista responsável contanto que atenda a todos os locais em que a coleta deve ser realizada. A figura 13 apresenta o percurso realizado referente à uma quinta-feira.
	
	De acordo com informações da SSM, a partir de 2019, após finalizada a coleta dos resíduos comuns às 16 horas (às segundas, terças, quintas e sextas-feiras,), o caminhão segue em direção ao aterro sanitário de Tremembé para a devida disposição final. O percurso realizado até o aterro sanitário é de aproximadamente 48 km (figura 16).
	
	Nesses dias em que o resíduo é encaminhado ao aterro sanitário, a média diária que o caminhão percorre é de 207,7 km para realização da coleta, transporte e retorno à base. Para execução da coleta o município dispõe de três funcionários, sendo um motorista e dois coletores.
	
	\subsubsection{Disposição final}
	
	O aterro sanitário de Tremembé pode receber resíduos Classe I, II A e II B. Ao chegar, o caminhão passa por uma balança rodoviária para aferição de sua massa e posteriormente o resíduo é encaminhado para o local de despejo onde é finalmente disposto em local adequado. Em seguida, o caminhão retorna à balança para aferir novamente sua massa.
	De uma forma geral, o sistema de coleta de resíduos comuns pode ser esquematizado conforme a figura 17.
	
	\subsubsection{Volume}
	
	Através da avaliação da série histórica dos dados cedidos pelo aterro sanitário de Tremembé, foi possível aferir algumas características existentes no município e sua dinâmica. O conhecimento desses dados é importante para entender o comportamento de geração de resíduos sólidos comuns, e assim elaborar ações e metas que auxiliem na melhoria da gestão dos resíduos.
	
	Nessa avaliação procurou-se conhecer a variação de geração do resíduo ao longo dos meses, onde foi possível perceber que, entre os anos avaliados, ocorreu uma tendência de maior geração entre os meses de dezembro e março, sendo janeiro o mês mais expressivo ao longo dos anos. Uma hipótese para explicar esse fato seria o acréscimo do número de turistas na região, no período.
	
	Há também um período de menor geração que ocorre entre agosto e setembro, conforme Figura 18. No apêndice A consta a geração pontual que cada segunda, quarta e sexta feira produzia ao longo dos meses comparativamente com as médias anuais e diárias ao longo dos anos avaliados (2014-2017).
	
	Nesta avaliação também foi possível obter e compreender como é a dinâmica de produção do resíduo comum pelos munícipes. Neste caso percebe-se que os dias que acontecem as coletas de maiores quantidades são segundas-feiras como pode ser vislumbrado na figura 19.
	
	Essa maior quantidade coletada às segundas possivelmente é em decorrência dos finais de semana, quando os munícipes ficam mais tempo em suas casas e também ocorre aumento do número de turistas na região. A geração de resíduos ao longo da semana tem uma dinâmica em que a produção em dias úteis é mais expressiva que a produção de resíduos nos finais de semana.
	
	A cada dia de coleta dos RSU comuns recolhem-se resíduos acumulados durante 2 a 3 dias pela população em Monteiro Lobato, como mostra a Tabela C.
	
	Outro ponto a ser destacado é a ocorrência da geração per capita diária de resíduos pelos munícipes e como ela se comporta ao longo dos anos observados. Para essa avaliação foram utilizadas as estimativas populacionais fornecidas pelo Instituto Brasileiro de Geografia Estatística (IBGE), bem como o total anual gerado. Nota-se que de 2015 até 2017, a geração per capta regride, como pode ser visto na figura 21.
	
	Analisando os dados foi possível verificar que a contribuição média diária para a geração de resíduos comuns é de 0,48 kg por habitante, totalizando em uma geração média de 4,44 toneladas por coleta e uma produção média mensal de 65,31 toneladas (2014-2017).
	
	\subsubsection{Caracterização gravimétrica} 	
	A caracterização gravimétrica tem como premissa entender a composição de todo o resíduo gerado em uma amostra, dividindo-o em categorias. Com este intuito, caracterizou-se o RSU comum no mês de dezembro de 2017.
	
	Para isso, foram utilizados uma balança com contrapeso e capacidade para 180 kg (+/-) 2 kg e um dinamômetro com capacidade para 50 kg (+/-) 0,050 kg para os resíduos mais leves (figura 22).
	
	Para a caracterização dos RSU, usou-se de forma adaptada para as características locais, a Norma NBR 10004/2004 na classificação e separação por tipo dos resíduos triados, a Norma NBR 1007/2004 na amostragem, e o Manual de Gestão dos Resíduos Sólidos do Instituto Brasileiro de Administração Municipal (IBAM, 2001) para o procedimento de gravimetria.
	
	No processo de caracterização gravimétrica foi observado que o caminhão compactador atualmente destinado ao recolhimento dos resíduos no município coleta aproximadamente 2/3 do que é gerado nas segundas e quartas-feiras, sendo, assim, necessária a complementação da coleta com outro caminhão. Este caminhão tem como função executar somente a coleta seletiva, no entanto, devido a problemas por falta de manutenção na frota, ambas as coletas são feitas pelo mesmo veículo.
	
	\paragraph{\textbf{Metodologia da caracterização gravimétrica:}}
	
	A caracterização dos resíduos da coleta comum foi realizada entre os dias 11 e 16 de dezembro de 2017. Em cada avaliação era utilizado material proveniente da coleta do dia anterior. Após a retirada das lixeiras públicas, o resíduo era transportado para um pátio da SSM onde ficava armazenado. O pátio possuía impermeabilização de cimento, porém não havia cobertura. No período da caracterização não houve incidência de chuva.
	
	Foram realizadas 3 amostragens, uma para cada dia de coleta semanal (antigamente realizada às segundas, quartas e sextas-feira). No dia seguinte a coleta, os resíduos armazenados eram espalhados pelo pátio, homogeneizados, quarteados e metade da amostra era removida com auxílio de uma escavadeira. O mesmo processo era aplicado na porção restante. Após, de forma manual, os sacos contendo resíduos eram abertos e o conteúdo novamente espalhado pelo pátio, homogeneizado e, posteriormente espalhados aleatoriamente de forma circular para retirada de 1 m$^{3}$ de resíduos.
	
	Para a pesagem dos resíduos foi utilizado um tambor cilíndrico de plástico de 0,2 m$^{3}$. O preenchimento do tambor foi feito recolhendo resíduos de 5 pontos distintos, percorrendo a circunferência. Sequencialmente, despejou-se novamente o conteúdo do tambor no chão e procedeu-se com a separação dos materiais por tipos e repesagem de cada tipo conforme pode ser observado na figura 24.
	
	Devido ao baixo efetivo e à impossibilidade de aferição de resíduos com menor massa, a divisão por tipos de RSU se limitou de acordo com a Tabela S.
	
	
	\paragraph{\textbf{Resultados da caracterização gravimétrica:}}
	
	Com os dados de cada amostra foi possível obter uma média da composição gravimétrica da coleta comum realizada no município, segundo a tabela 4. No final de semana antecedente à caracterização gravimétrica, houve uma forte chuva no município, e como os resíduos ficam dispostos nas lixeiras coletivas para posterior coleta, a composição gravimétrica da coleta de segunda-feira pode ter sofrido variação na massa em decorrência do volume de água envolvido juntamente aos resíduos.
		
	Foram obtidas as densidades aparente dos resíduos e sua média referentes aos dias de avaliação, como pode ser observado na tabela 5. A densidade média é um valor que pode ser utilizado para o dimensionamento de equipamentos de transporte, instalações de transbordo e tratamento de resíduos sólidos (MONTEIRO, 2001).
	
	A média de geração dos RSU comuns referente aos dias de pico avaliados encontra-se ilustrada na Figura 25. O resultado caracteriza-se pelo maior conteúdo de material orgânico, atingindo 34,1 \% do total do RSU. Em seguida, com 13 \% e 12,8 \% respectivamente, ocorre a geração de materiais higiênicos e tecidos, sendo em sua maioria roupas ainda em bom estado (figura 26). 
	
	Os plásticos como um todo somam 18 \% do RSU gerado, sendo 11,4 \% de plástico fino e 6,6 \% de plástico duro; 9,9 \% do total do RSU são constituídos de papel e papelão; 5,0 \% são constituídos de metais ferrosos, não ferroso e alumínio; 2,7 \% por madeiras tratadas e não tratadas (naturais); 1,7 \% por pneus; 1,4 \% por vidros coloridos e transparentes; e 1,4 \% por outros materiais que são passíveis de logística reversa e  materiais de RSS que devem ser acondicionados e armazenados diferenciadamente para posterior tratamento (seringas, medicamentos e embalagens medicamentosas).
	
	Para obtenção de um resultado mais próximo da realidade do município, faz-se necessária outra caracterização em um período de baixa geração ou menor sazonalidade.
	
	\subsubsection{Custos}
	
	A equipe responsável pelos serviços de coleta de resíduos sólidos comuns é composta por três funcionários, sendo dois coletores e um motorista. De acordo com documento fornecido pela SSM em 2017, considerando os custos de salário base, descontos e FGTS, estima-se que sejam destinados de remuneração, R$ 1.841,08 para cada funcionário. Assim, o valor total é de R$ 5.523,24 para os três funcionários.
	
	Em relação ao custo destinado à manutenção de equipamentos para o serviço de coleta, em 2017, a SSM declarou ser gasto uma média de R\$ 1.800,00 em manutenção de equipamentos do serviço de coleta comum.
	
	Em relação ao transporte da coleta comum, a estimativa do gasto foi calculada considerando a média da quilometragem da rota percorrida para coleta, transporte e disposição final até o aterro sanitário às segundas, terças, quintas e sextas-feiras (207,7 km por dia). 
	
	Considerando o preço do litro do diesel disponibilizado pela SSM de Monteiro Lobato (XXX) e a quilometragem total rodada para os quatro dias (830,8 km), o gasto referente ao transporte do serviço de coleta comum é de, aproximadamente, R\$ 3.509,11. A tabela 6 apresenta o resumo dos custos associados ao serviço público de coleta de resíduo comum de Monteiro Lobato.
	
	O custo associado ao serviço de disposição final do resíduo comum de Monteiro Lobato realizado no aterro sanitário ESTRE-Resicontrol foi obtido através da solicitação dos contratos desde a validação do primeiro contrato com a Secretaria de Finanças do município. A tabela 7 apresenta o histórico de custo associado ao serviço de disposição final do resíduo comum coletado pelo município do período de 2014 a 2017, considerando que são realizados contratos anuais com a ESTRE-Resicontrol.
	
	\subsection{Resíduos da Coleta Seletiva}
	
	Através da obtenção dos dados referentes a geração e coleta de resíduos sólidos recicláveis da plataforma Série Histórica do SNIS para o ano de 2015, a tabela 8 foi construída (SNIS, 2009 a 2015). Analisando a tabela, observa-se que a taxa de cobertura de coleta seletiva porta-a-porta do país abrange apenas metade da população urbana e que este valor não é homogêneo para cada região.
	
	Enquanto a região Sul possui a maior taxa de coleta seletiva, abrangendo 77 \% da população urbana, as regiões Norte e Nordeste apresentam taxa de coleta seletiva em torno de apenas 15 \%. As regiões Centro-Oeste e Sudeste são as que mais se assemelham a taxa de cobertura do país.
	
	A região Sul se destaca em relação aos outros parâmetros analisados de taxa de material recolhido em relação ao total de resíduos sólidos domésticos (exceto matéria orgânica e rejeitos), massa per capita de materiais recicláveis coletados e massa recuperada em relação a população urbana.
	
	Os valores da tabela 8 indicam que, em geral, o Brasil ainda conta com um sistema precário de coleta seletiva e recuperação dos materiais recolhidos. Para melhorar esse panorama, a taxa de cobertura de coleta seletiva deveria ser mais abrangente e, para que o sistema fosse efetivo, seria necessário maior investimento em educação ambiental para que a população fosse mobilizada e sensibilizada a fazer a correta segregação do resíduo domiciliar. Através dessas ações, a taxa de RSU destinada a aterros sanitários pode diminuir consideravelmente.
	
	Os dados primários sobre os resíduos recicláveis em Monteiro Lobato foram obtidos por meio de questionário dirigido às SMAA e SSM e avaliação presencial da coleta municipal.
	
	\subsubsection{Acondicionamento}
	
	Os resíduos recicláveis são acondicionados de forma semelhante à dos resíduos sólidos comuns, em sacos plásticos ou em sacolas de supermercados. Da mesma forma, o armazenamento destes resíduos é semelhante ao dos resíduos sólidos comuns, sendo pendurados em portões ou dispostos nas lixeiras de ferro ou de madeira.
	
	Além disso, resíduos maiores como caixas de papelão e isopor costumam ser acondicionados sem nenhuma forma de revestimento nas lixeiras, como apresentado na figura 27.
	
	\subsubsection{Coleta, Transbordo e Transporte}
	
	Um problema pelo qual o município passa é em relação ao armazenamento e coleta dos resíduos recicláveis. Atualmente, a coleta do resíduo reciclável ocorre às quartas-feiras em período integral, das 7h às 16h.
	Como o município não dispõe de lixeiras específicas para resíduos comuns e resíduos recicláveis ou de estratégias para diferenciar o acondicionamento, os sacos plásticos de resíduos comuns e recicláveis ficam misturados nas lixeiras e os coletores não são capazes de diferenciar as composições dos sacos no momento da coleta. Deste modo, durante as coletas dos resíduos recicláveis são coletados apenas os resíduos de fácil identificação, como caixas de papelão e garrafas, como exemplificado na figura 28.
	
	O serviço de coleta de resíduos sólidos recicláveis é realizado pelos mesmos três funcionários responsáveis pela coleta de resíduos comuns. A coleta dos recicláveis ocorre através de um caminhão tipo compactador exclusivo com capacidade de 15 m$^{3}$(vide figura 29), que percorre todo o município e encaminha os resíduos para Jacareí.
		
	Conforme descrição da SSM, a coleta de resíduos sólidos recicláveis passa pelas mesmas 59 lixeiras nas quais são armazenados os resíduos sólidos comuns, cujas localizações foram discutidas para o resíduo sólido comum.
	O processo de retirada dos resíduos recicláveis é o mesmo que o realizado para os resíduos sólidos comuns. É feita uma manobra com o caminhão aproximando-o de ré até próximo a lixeira como ilustra a figura 30. Nesta figura é possível visualizar a dificuldade dos coletores para coleta dos resíduos que não são adequadamente acondicionados.
	
	Como descrito anteriormente, na segunda semana de dezembro de 2017 os percursos de coleta foram acompanhados e as rotas foram registradas através do GPS. Nesse ano, a coleta seletiva costumava ocorrer às terças e quintas-feiras no período da tarde, ao invés de ser às quartas-feiras como ocorre atualmente. Na Figura 32, é apresentada a rota realizada para a coleta seletiva, referente a uma quinta-feira, mas que pode ser refletida a rota atualmente realizada.
	
	Ao término de toda a coleta, é feito um percurso até a URBAM em São José dos Campos, para que seja efetuada a triagem deste material nas instalações do aterro. De acordo com os dados disponibilizados pela SSM, a quilometragem média percorrida para realização da coleta dos resíduos sólidos recicláveis no município e armazenamento na garagem da SSM é de 20 km. No entanto, no terceiro dia de coleta de recicláveis, os resíduos acumulados são encaminhados à URBAM e o caminhão retorna à base, a uma distância média percorrida de 123,3 km, levando em consideração que a distância entre Monteiro Lobato e a URBAM é de aproximadamente 48 km. A localização da URBAM em relação ao município é apresentada na figura 33.
	
	\subsubsection{Disposição final}
	
	Como descrito anteriormente, na terceira coleta de resíduos sólidos recicláveis, o caminhão encaminha os resíduos acumulados até a Urbanizadora Municipal de São José dos Campos – URBAM.
	
	Quando o caminhão chega à URBAM, os resíduos são encaminhados à área de triagem. Após a deposição do material na área de triagem da URBAM, os resíduos passam por uma esteira rotativa sendo separados manualmente por catação, do início ao fim do processo. Os materiais selecionados e separadas são enfardados para então serem comercializado, enquanto os materiais descartados na triagem, serão destinados a área de aterramento onde receberão sua disposição final.
	
	De uma forma geral, o sistema de coleta de resíduos seletivos pode ser esquematizado conforme a figura 34.
	
	\subsubsection{Volume gerado}
	
	Através de dados disponibilizados pela SSM, dos valores totais mensais e a quantidade de coletas realizadas dos meses de março a outubro de 2017, foi possível realizar algumas análises sobre a dinâmica e atuação do município para com os resíduos recicláveis. A avaliação dessas informações permite que sejam compreendidos a dinâmica de geração de resíduos recicláveis e os hábitos de segregação entre resíduos recicláveis e comuns e auxilia no desenvolvimento de diretrizes e ações que visem a melhoria do gerenciamento dos resíduos recicláveis.
	
	Com base nos dados fornecidos e projeções populacionais do IBGE foi possível estimar a geração per capita diária de resíduos sólidos recicláveis entre março e outubro de 2017 (vide figura 35). Percebe-se uma tendência entre os meses de março e agosto, que apresentam valor per capita diário de aproximadamente 0,04 kg por habitante por dia. Para os meses de setembro e outubro é observada uma queda brusca na geração.
	
	Com a estimativa desses dados é possível verificar que o potencial de reciclagem do município é muito baixo. Considerando que a geração per capita de resíduo sólido domiciliar de Monteiro Lobato é a soma entre a geração per capita diária de resíduo comum e a geração per capita diária de resíduo reciclável, a geração per capita de resíduo reciclável representa, aproximadamente, 8 \% da geração per capita total.
	Na coleta dos RSU seletivos, recolhe-se todo o resíduo gerado pela população durante uma semana anterior. Um problema recorrente no município é a ausência de lixeiras separadas para resíduos recicláveis, o que pode acarretar em uma diferença na quantidade de resíduos recicláveis coletados para a quantidade real de resíduos colocados nas lixeiras além de haver dificuldade na diferenciação dos sacos de resíduos recicláveis dos comuns pelos coletores.
	
	\subsubsection{Caracterização gravimétrica}
	
	\paragraph{Metodologia da caracterização gravimétrica}
	
	A caracterização gravimétrica e a avaliação dos RSU seletivos ocorreram igualmente ao processo descrito para a metodologia dos resíduos sólidos comuns (item 1.2.5). Para os resíduos provenientes da coleta seletiva foram realizadas 2 amostragens dos seus respectivos dias de coleta semanal (terça e quinta-feira, dias antigos de coleta reciclável referente à data em que essa caracterização foi realizada, em 2017). Devido ao baixo efetivo e à impossibilidade de aferição de resíduos com menor massa, a divisão por tipos de resíduos se limitou à disposta na tabela 3. Não houveram alterações nas condições climáticas que comprometessem a caracterização desses resíduos.
	
	\paragraph{Resultados da caracterização gravimétrica}
	
	Com os dados de cada amostra foi possível obter uma média da composição gravimétrica da coleta comum realizada no município, conforme a tabela 10.
	
	Medindo-se as amostras obteve-se as densidades aparentes dos resíduos da coleta seletiva referentes aos dias de avaliação, bem como a média como pode ser observado na tabela 11.
	
	A característica dos RSU seletivos (figura 36) no período de maior geração no município é de 40,5 \% de papelão como material presente em maior quantidade seguido de plásticos totalizando 25,1 \% sendo que desses, 14,4 \% são plásticos duros e 10,7 \% são plásticos finos. Vidros coloridos e transparentes contribuem com 19,4 \% e papeis representam 3,8 \%, seguidos pelos materiais que são passíveis de logística reversa (embalagens de óleos automotivos, lâmpadas fluorescentes, equipamento eletroeletrônicos e pilhas) e materiais de RSS (seringas, medicamentos e embalagens medicamentosas), que devem ser acondicionados e armazenados diferenciadamente para posterior tratamento, constituindo estas classes 2,4 \% do total. 
	Na caracterização gravimétrica também foi catalogada a presença de materiais higiênicos (2,3 \%) e materiais orgânicos (2,1 \%), os quais contaminam materiais recicláveis, podendo reduzir ou até inviabilizar a reciclagem desses materiais com tal potencial. Em menor proporção ocorreram embalagens Tetra Pak com 1,3 \%, metais diversos 1,1 \%, isopor 0,5 \% e tecido 1,4 \% conforme demonstra a figura 36.
	
	Para um resultado mais próximo da realidade do município, faz-se necessária outra caracterização em um período de baixa geração ou menor sazonalidade.
	
	\subsubsection{Custos}
	
	A distribuição dos custos relacionados aos resíduos recicláveis é similar aos custos referentes aos resíduos comuns. Os funcionários responsáveis pelo serviço de coleta de resíduos recicláveis são os mesmos para a coleta de resíduo comum. Desta forma, o salário destinado aos três funcionários engloba o serviço de coleta de resíduos comum e reciclável.
	De acordo com dados fornecidos pela SSM, o valor destinado à manutenção de equipamentos para este serviço é R\$ 1.800,00 por mês. 
	
	Em relação ao transporte, a estimativa do valor gasto foi a mesma que para a despesa relacionada ao resíduo sólido comum. O cálculo foi feito considerando a média da quilometragem da rota percorrida apenas para coleta dos resíduos (20 km) e a média da quilometragem da coleta, transporte e destinação final até a URBAM, quando realizada a terceira coleta de resíduos recicláveis (123,3 km).
	
	Considerando o preço do litro do diesel disponibilizado pela SSM de Monteiro Lobato e a quilometragem total rodada, o gasto referente ao transporte do serviço de coleta reciclável é de, aproximadamente, R\$ 438,70. A tabela 12 apresenta o resumo dos custos associados ao serviço público de coleta de resíduo comum de Monteiro Lobato.

	Como o município não apresenta contrato firmado com a URBAM de São José dos Campos, não existe um controle sobre o valor do serviço prestado pela disposição final.
	
	\subsection{Resíduos de Limpeza Urbana}
	
	Os resíduos sólidos do serviço de limpeza urbana englobam os oriundos de varrição, limpeza de logradouros e vias públicas, poda de árvores, capina e roçagem, limpeza de feiras, limpeza de praias, limpeza da rede de drenagem, limpeza de bocas de lobo e desobstrução de ramais (BRASIL, 2010a; BRASIL, 2007a; IBAM, 2001) e o seu manejo objetivam manter as condições de higiene dos logradouros e a boa qualidade dos sistemas de drenagem de águas pluviais evitando assim problemas de saúde de caráter sanitário e inundações nas regiões urbanas. Os serviços de limpeza urbana também evitam a geração de maus cheiros oriundos da decomposição de matéria orgânica de frutos ou de materiais provenientes de feiras. Em Monteiro Lobato são realizadas as atividades de varrição das vias públicas, poda e capina, limpeza de sistema de drenagem pluvial, limpeza de cemitério e de feira livre.
	
	De acordo com o artigo 7 da Lei nº 11.445 de 2007, o serviço público de limpeza urbana deve se responsabilizar pela coleta, transbordo e transporte dos resíduos; pela respectiva triagem para fins de reuso ou reciclagem; pelo tratamento, inclusive por compostagem, e disposição final dos resíduos originários dos serviços de limpeza urbana descritos anteriormente (BRASIL, 2007a). Uma boa atuação da Prefeitura nesse sentido pode refletir nas atitudes dos munícipes incentivando a correta destinação e desenvolver o grau de educação ambiental. Além disso, uma satisfatória limpeza urbana promove uma estética agradável para a cidade, o que é vantajoso para Monteiro Lobato pois se trata de um município que investe no desenvolvimento turístico.
	 
	As informações sobre manejo da limpeza urbana, custos associados, equipamentos utilizados, responsabilidades dos funcionários e horários da realização dos serviços foram obtidas através de questionário estruturado encaminhado em forma de ofício à SSM e SMAA. De acordo com as respostas, além da quantidade específica de funcionários para atividades de varrição e poda e capina, existem sete funcionários (cinco de frente de trabalho, um da SMAA e um da SSM) que colaboram com as demais atividades referentes à limpeza urbana, quando necessário.
	
	O serviço de varrição em Monteiro Lobato ocorre todos os dias da semana, das 07h às 11h e das 12h às 16h. Entre segunda e sexta-feira são efetuadas a limpezas nas praças e logradouros da região central, rodoviária e nos bairros Jardim Iracema, Vila São Sebastião, Jardim Morada do Sol, Vila Esperança, Souza e São Sebastião. Aos finais de semana a limpeza fica limitada ao serviço a região central e rodoviária no período da manhã.
	
	Durante o serviço de varrição, ocasionalmente os sacos plásticos de 60 litros das lixeiras públicas são retirados, colocados no carrinho de mão, encaminhados e armazenados no pátio da SSM para posteriormente serem coletados pelo caminhão da coleta comum.
	
	Para a varrição, o município dispõe de cinco funcionários contratados pela prefeitura e de, normalmente, cinco funcionários da frente de trabalho, que colaboram com o serviço. Para a varrição, são utilizadas vassouras, pás de lixo e carrinho de lixo de capacidade de 240 litros (vide figura 37), sendo que cada funcionário percorre em média 3 km por dia.
	
	Outro serviço de limpeza urbana prestado pelo município é o de poda e capina e que ocorre de duas formas distintas. Às sextas-feiras ocorrem as podas, conforme pedido efetuado pelo munícipe à Central do Cidadão. Já de segunda a sexta-feira ocorre a capina conforme cronograma da SSM. A tarefa é normalmente realizada por quatro funcionários. Os equipamentos utilizados para poda e capina são a enxada, enxadão, pá, motosserra, foice, machado, serra manual e caminhão basculante com capacidade para 5 metros cúbicos.
	
	Em Monteiro Lobato é realizada uma feira semanalmente aos sábados, das 9 horas às 14 horas, onde comercializa-se principalmente produtos alimentícios da própria região. A feira é de pequeno porte e seus resíduos gerados são acondicionados nas lixeiras públicas pelos próprios munícipes. Semanalmente, ocorre a limpeza do cemitério municipal que produz como resíduo, em sua maioria, flores, vasos e grãos provenientes da varrição (figura 38). Este serviço é conduzido por um funcionário.
	
	Os resíduos da rede de drenagem são provenientes da limpeza de bocas de lobo, sarjeta e bueiros, de onde se retiram os sedimentos como areia e argila e resíduos sólidos que se depositam nestes locais prejudicando a drenagem das águas pluviais. Essa limpeza ocorre semanalmente e quando há solicitação pelos moradores através da Central do Cidadão. O trabalho usualmente é realizado normalmente por quatro funcionários, através de enxadas, pás e carrinho de mão de capacidade de 60 litros, sendo três funcionários para limpeza e um motorista.
	
	\subsubsection{Acondicionamento}
	
	Após as varrições, os resíduos acumulados são acondicionados em sacos plásticos e encaminhados e armazenados na garagem da SSM.  Já os resíduos provenientes do serviço de poda e capina são armazenados no pátio da Prefeitura localizado no Bairro Morada do Sol, apresentado na figura 39, até serem coletados e destinados. Os resíduos oriundos do serviço de limpeza da rede de drenagem são acondicionados em carrinhos de mão e são também armazenados no pátio do bairro Morada do Sol até coleta.
	
	Os resíduos gerados pela feira livre do município são coletados pelos próprios usuários da feira e são acondicionados em sacos pretos de 200 litros dentro das lixeiras públicas, como as da figura 40, localizadas próximas aos mercados da praça.
	
	Os resíduos do cemitério são acondicionados em sacos plásticos de 200 litros e mantidos no próprio cemitério até que seja realizada a coleta.
	
	\subsubsection{Coleta, Transbordo e Transporte}
	
	Após os respectivos acondicionamentos dos resíduos sólidos oriundos da limpeza urbana, todos sacos plásticos são coletados pelo caminhão responsável pela coleta comum do município e seguem para o mesmo destino dos resíduos sólidos domiciliares, o aterro sanitário de Tremembé ESTRE-Resicontrol.
	
	\subsubsection{Disposição final}
	O material proveniente do serviço de poda e capina, atualmente, é encaminhado para o pátio do (CDM) no bairro Morada do Sol. Neste local ocorre a compostagem aeróbica para transformação do material orgânico em composto, que é usado em atividade de jardinagem. Como esse procedimento é novo no município, não se tem a quantidade de material encaminhado e transformado neste local. 
	A disposição final dos resíduos oriundos do serviço de varrição, limpeza de rede de drenagem, feiras e do cemitério é realizada em conjunto com os resíduos sólidos comuns do município de Monteiro Lobato, no aterro sanitário ESTRE-Resicontrol.
	
	\subsubsection{Volume}
	
	O município não possui um controle da quantidade gerada em volume ou em peso dos resíduos oriundos dos serviços de limpeza urbana de varrição, poda e capina, limpeza do cemitério, limpeza da feira e limpeza da rede de drenagem.
	Apesar de se conhecer o volume dos sacos plásticos onde os resíduos são acondicionados, não há o controle da quantidade de sacos plásticos que costumam ser retirados para cada atividade ou se os mesmos são preenchidos até a sua capacidade total.
	
	\subsubsection{Custos}
	
	Como o município não possui renda específica para gerenciamento dos resíduos sólidos e a manutenção dos equipamentos do serviço de limpeza urbana não é frequente não há um controle sobre o custo associado a esse serviço. No entanto, é possível estimar o custo relacionados aos funcionários para cada atividade considerando o valor médio de salário base somado aos descontos e FGTS calculado através dos dados disponibilizados pela SSM e pela quantidade de funcionários responsáveis por cada serviço. A estimativa do gasto da Prefeitura com os serviços de limpeza urbana engloba também o valor destinado aos cinco funcionários de frente de trabalho que colaboram com a manutenção e atividades em alguns serviços, ajudando principalmente na varrição.  A tabela 13 apresenta os gastos aproximados com os funcionários para cada atividade de limpeza urbana, obtidos em 2017.
	
	
	\subsection{Resíduos dos Serviços Públicos de Saneamento Básico}
	
	Resíduos de serviços públicos e saneamento básico são aqueles englobados por um conjunto de serviços, infraestruturas e instalações operacionais de: abastecimento de água potável, esgotamento sanitário, limpeza urbana e manejo de resíduos sólidos e drenagem e manejo das águas pluviais, limpeza de vias públicas e logradouros, limpeza e fiscalização preventiva das respectivas redes urbanas (BRASIL, 2010a).  
	
	As redes de drenagem urbana tradicionalmente são compostas por dois subsistemas: o sistema inicial de drenagem e o sistema de macrodrenagem. O primeiro é composto pelos pavimentos das ruas, guias e sarjetas, bocas de lobo, rede de galerias de águas pluviais e canais de pequenas dimensões. Esse sistema, quando bem planejado e mantido, contém as enxurradas e inundações, evitando problemas nas atividades urbanas. O sistema de macrodrenagem é composto por canais, abertos ou de contorno fechado, de dimensões maiores que os do sistema inicial de drenagem (PREFEITURA MUNICIPAL DE SÃO PAULO, 1999).
	
	As redes de drenagem urbana são uma fonte de poluição difusa e principal fato degradante de rios, lagos e estuários pois veiculam uma grande e diversa carga de poluentes provenientes da disposição incorreta dos mesmos na superfície (BRITES et al, 2003). Na rede de microdrenagem do município de Monteiro Lobato são lançados materiais de diversos tipos, obstruindo a passagem das águas pluviais e assim tornando-se um fator favorável às inundações e alagamentos (PMML, 2007).
	
	Os titulares dos serviços públicos de saneamento básico poderão delegar a organização, regulação, a fiscalização e a prestação desses serviços. Cabe ao titular formular a política pública de saneamento básico através da elaboração de planos de saneamento básico, dentre outras atividades previstas na Lei nº 11.445 de 2007.
	
	É de competência do Estado de São Paulo o planejamento, fiscalização, regulação dos serviços municipais de abastecimento de água e esgotamento sanitário no município. Já a limpeza, desobstrução e recolhimento de detritos formados nas bocas de lobo, que compõem a rede de microdrenagem do município são de responsabilidade do município. O serviço é realizado semanalmente pela SSM ou quando solicitado por munícipes através de requerimento via Central do Cidadão.
	
	No município de Monteiro Lobato, a Lei 1.378 de 10 de outubro de 2007 autoriza a Companhia de Saneamento Básico do Estado de São Paulo (SABESP) a realizar os serviços municipais de abastecimento de água e esgotamento sanitário. Esses serviços abrangem, no todo ou em parte, as atividades de captação, adução e tratamento de água bruta; a adução, reserva e distribuição de água tratada; e a coleta, transporte, tratamento e disposição final de esgotos sanitários (MONTEIRO LOBATO (SP), 2007).

	Neste contexto, os lodos gerados das estações de tratamento de água (ETA) e estações de tratamento de esgoto (ETE) são classificados como resíduos sólidos pela norma da Associação Brasileira de Normas Técnicas (ABNT) NBR 10004:2004 e podem ser englobados como resíduos de serviços públicos e saneamento básico (GOVERNO DO ESTADO DE SÃO PAULO, 2014).
	
	\paragraph{\textbf{Sistema de ETA em Monteiro Lobato}}
	
	O município é composto por três sistemas de abastecimento de água: O principal, que é denominado Sistema Sede, e os isolados, determinados de Reservatório São Benedito e ETA Jair Dimas, no bairro do Souzas, apresentados nas figuras 41, 42 e 43. 
	
	Em 2007 estes sistemas atendiam 99\% da área urbana (PMML, 2007). Atualmente, de acordo com a SABESP, a ETA Sede trata 10.686 m$^{3}$ de água por mês e a ETA do Bairro do Souzas, trata um volume mensal de 1.795 m$^{3}$. Não foi informado o volume de água tratada por mês pelo Reservatório São Benedito, que recebe a água do poço. Como a água do poço contém ferro e manganês, esta água passa por um filtro. Atualmente este filtro é limpado diariamente e os resíduos retirados são descartados diretamente no corpo receptor ao redor.
	
	\paragraph{\textbf{Sistema de ETE em Monteiro Lobato}}
	
	O sistema de esgotamento sanitário do município de Monteiro Lobato abrange a área urbana do município e contempla de 4 estações, sendo elas: Sistema Sede, ETE Jardim Iracema, ETE Souzas e ETE São Benedito. Em 2007, apresentava um índice de coleta de 73 \% e tratamento de 88 \% desse esgoto coletado. 
	O Sistema Sede possui uma e capacidade nominal de 6 l/s trata 5,6 l/s de esgoto da área urbana e o efluente final tratado é lançado no Rio Buquira. A ETE Jardim Iracema se encontrava em reforma durante o levantamento dos dados e não foi possível realizar a visita. 
	O Sistema da ETE do Bairro dos Souzas na Figura 44, possui apenas um sistema coletor isolado, contempla 50 \% do bairro e lança o esgoto in natura no Córrego Faria (PMML, 2007). 
	
	A ETE São Benedito trata o esgoto que é recebido de uma Estação Elevatório, como mostra, respectivamente, na figura 46. 
	
	Tanto a ETE São Benedito quanto a ETE do Bairro do Souzas apresentam um sistema preliminar composto por gradeamento e desarenador, como apresentado na figura 47. Os sólidos grosseiros retidos nas grades são retirados periodicamente, acondicionados em baldes e levados para o Sistema Sede, onde são acondicionados em tanques juntamente com os resíduos provenientes dos filtros. Estes resíduos são coletados por empresa terceirizada.
	
	\subsubsection{Disposição final}
	
	A empresa responsável pela coleta do lodo e do resíduo gerado na etapa de gradeamento no Sistema de Tratamento de Esgoto é a Essencis Ecossistema Ltda, que atua, nesse serviço, desde 2010 no município. Além disso, a empresa é responsável por coletar os resíduos oriundos dos desarenadores e os resíduos retidos no Elevatório São Benedito (vide figura 48). 
	
	A coleta dos resíduos procedentes das ETAs e ETEs ocorre bimestralmente. A coleta da empresa abrange todas as subestações, com exceção do Reservatório São Benedito e da ETA do Bairro do Souzas, que atualmente também descartam o lodo gerado em corpo receptor, mas tem previsões de instalar um tratamento ou leito de secagem.
	
	Segundo a SABESP, atualmente, o lodo coletado no município é levado para a ETE Jardim Iracema onde ocorre o processo de tratamento e o lodo ativado é disposto em leito de secagem. Passado o tempo necessário, o lodo é acondicionado em caçambas para ser feita a destinação final
	
	Assim, tanto o lodo gerado no tratamento do esgoto sanitário quanto os resíduos provenientes do gradeamento das estações são destinados ao aterro sanitário certificado para sua disposição final, de acordo com informações concedidas pela SABESP. 
	
	\subsubsection{Volume}
	De acordo com informações levantadas pela SABESP, a ETE Jardim Iracema com seu processo de tratamento do esgoto sanitário tem uma geração de lodo seco desidratado de 10 toneladas por ano.
	
	\subsection{Resíduos Industriais}
	
	Os resíduos industriais são aqueles gerados nos processos produtivos e de instalações industriais cujas particularidades tornem inviável o seu lançamento na rede pública de esgoto ou em corpos d’água, ou exijam, para isso, soluções técnica ou economicamente inviáveis (BRASIL, 2010a; CONAMA, 2002a). Podem ser originados dos mais diversos ramos industriais, como metalúrgico, químico, petroquímico, alimentício, mineração, entre outros (IPEA, 2012b).
	
	Devido às diferentes possibilidades de origem, os resíduos industriais podem ser compostos por resíduos perigosos ou não perigosos (CONAMA, 2002a). Os resíduos industriais abrangem resíduos de processo, resíduos de operação de controle de poluição ou descontaminação, resíduos da purificação de matérias-primas e produtos, cinzas, lodos, óleos, resíduos alcalinos, resíduos ácidos e até mesmo resíduos plásticos, papel, madeira, fibras, borracha, metal, escórias, vidros e cerâmicas (IPEA, 2012b).
	
	No Brasil há 4 unidades de aterros industriais cadastradas, sendo 3 da região sudeste e 1 da região sul. Dessas, 1 é operada por instituição pública e 3 por empresas privadas. A massa total de resíduos recebidos por essas unidades de processamento é de 5.793 toneladas (SNIS, 2017).
	
	Em escala estadual, foi previsto um aumento da geração de resíduos industriais de cerca de 95,8 milhões de t/ano, em 2010, para próximo de 190,7 milhões t/ano em 2030. A tabela 14 mostra a tendência de crescimento na geração dos resíduos industriais entre 2010 e 2030 (GOVERNO DO ESTADO DE SÃO PAULO, 2014).
	
	O município de Monteiro Lobato tem apenas uma planta industrial, denominada “Água Mineral Natural Monteiro Lobato da Mineração Monteiro Lobato Ltda”. Através de visita à essa planta em conjunto com a SMAA (vide figura 49), foi verificado que a planta apresenta Plano de Gerenciamento de Resíduos Sólidos, segue os regulamentos do Departamento Nacional de Produção Mineral (DNPM), está devidamente licenciada para operação com a Companhia Ambiental do Estado de São Paulo (CETESB) e regularizada com o Cadastro Técnico Federal de Regularidade do Instituto Brasileiro do Meio Ambiente e dos Recursos Naturais Renováveis (IBAMA).
	
	\subsubsection{Diretrizes iniciais para elaboração do Plano de Gerenciamento de Resíduos Sólidos Industriais}
	
	Considerando os princípios da gestão integrada e compartilhada da PNRS, e de acordo com Plano de Resíduos Sólidos do Estado de São Paulo são atribuídas as seguintes responsabilidades aos geradores de resíduos industriais:
	
	\begin{itemize}
		\item o gerenciamento desde a geração até a disposição final do resíduo sólido e elaboração de Plano de Gerenciamento de Resíduos Sólidos.
		\item para geradores de resíduos industriais perigosos - o gerenciamento, mesmo que tratados, reciclados ou recuperados para utilização como adubo, matéria prima ou fonte de energia, bem como no caso de duas incorporações em materiais, substâncias ou produtos (dependerá de prévia aprovação dos órgãos competentes) e elaboração de Plano de Gerenciamento de Resíduos Sólidos.
	\end{itemize}

	De acordo com o Art. 13 da PNRS, os resíduos industriais caracterizados como não perigosos podem ser considerados similares aos resíduos domiciliares diante de sua natureza e composição pelo poder público municipal. No entanto, o plano de gerenciamento de resíduos sólidos industriais deve ser elaborado pelo gerador independentemente da composição do resíduo gerado (BRASIL, 2010a).

	Além de seguir a ordem de prioridade de não geração, redução, reutilização, reciclagem, tratamento e destinação final, de acordo com o Art. 21 da PNRS, o conteúdo mínimo que um plano de gerenciamento de resíduos sólidos deve apresentar engloba (BRASIL, 2010a):
	
	\begin{enumerate}[label=\Roman*]
%o uso do parênteses na atribuição do rótulo coloca a numeração da	lista entre parênteses. Se não usado, não exibe parênteses		
		\item Descrição do empreendimento ou atividade;
		\item Diagnóstico dos resíduos sólidos gerados ou administrados contemplando a origem, o volume e a caracterização dos resíduos, incluindo os passivos ambientais a eles relacionados;
		\item Observadas as normas estabelecidas pelos órgãos do Sisnama, do SNVS e do Sistema Unificado de Atenção à Sanidade Agropecuária (SUASA) e, se houver, o plano municipal de gestão integrada de resíduos sólidos:
		\begin{enumerate}[label=(\alph*)] 
			\item Explicitação dos responsáveis de cada etapa do gerenciamento de resíduos sólidos;
			\item Definição dos procedimentos operacionais relativos às respectivas etapas do gerenciamento de resíduos sólidos;
		\end{enumerate}
		\item Identificação das soluções consorciadas ou compartilhadas com outros geradores;
		\item Ações preventivas e corretivas a serem executadas em situações de gerenciamento inadequado ou acidentes;
		\item Metas e procedimentos relacionados à minimização da geração de resíduos sólidos e, de acordo com as normas estabelecidas pelos órgãos do Sisnama, do SNVS e do SUASA, à reutilização e reciclagem;
		\item Ações relativas à responsabilidade compartilhada pelo ciclo de vida dos produtos, na forma do artigo 31 da PNRS, se couber;
		\item Medidas saneadoras dos passivos ambientais relacionados aos resíduos sólidos gerados;
		\item Periodicidade de sua revisão, considerando, se couber, o prazo de vigência da respectiva licença de operação a cargo dos órgãos do Sisnama.
	\end{enumerate}

	Vale ressaltar que a elaboração e aplicação do plano de gerenciamento de resíduos sólidos não é afetada ou impedida diante da ausência do PMGIRS (BRASIL, 2010a).

	\subsubsection{Inventário de Resíduos Sólidos Industriais}
	
	O Inventário Nacional de Resíduos Sólidos Industriais, instrumento da PNRS, apresenta informações sobre geração, armazenamento, transporte, tratamento, reutilização, reciclagem, recuperação e disposição final destes resíduos. 
	
	Para alguns setores estas informações deveriam ser apresentadas até o ano de 2003, sendo eles: indústrias de preparação de couros e fabricação de artefatos de couro; fabricação de coque, refino de petróleo, elaboração de combustíveis nucleares e produção de álcool; fabricação de produtos químicos; metalurgia básica; fabricação de produtos de metal; fabricação de máquinas e equipamentos, máquinas e equipamentos, máquinas para escritório e equipamentos de informática; fabricação e montagem de veículos automotores, reboques e carrocerias; e fabricação de outros equipamentos de transporte. Essas informações devem ser atualizadas a cada dois anos (CONAMA, 2002a).
	
	Apesar da obrigatoriedade, não foi identificado o Inventário de Resíduos Sólidos Industriais do estado de São Paulo. Esse fato intensifica a necessidade da elaboração de planos de gerenciamento de resíduos industriais e da devida fiscalização pelo órgão ambiental responsável para que os RI do estado detenham disposição final ambientalmente adequada.
	
	As normas constituem uma ferramenta importante para a correta classificação de resíduos industriais e gerenciamento dos mesmos. Estas são listadas na tabela 15.
	
	\subsubsection{Volume}
	De acordo com as atividades da planta industrial, a geração de resíduos se dá principalmente por resíduos sólidos semelhantes aos resíduos sólidos domiciliares, e por resíduos sólidos oriundos da operação como plásticos, óleos de maquinários, lâmpadas e rejeitos.
	
	
	\subsection{Resíduos de Serviços de Saúde}
	Os RSS são definidos como os resíduos provenientes de atividades de atendimento à saúde humana ou animal, incluindo os serviços de assistência domiciliar e de trabalhos de campo; laboratórios analíticos de produtos para saúde; necrotérios, funerárias e serviços onde se realizem atividades de embalsamamento; serviços de medicina legal; drogarias e farmácias inclusive as de manipulação; estabelecimentos de ensino e pesquisa na área de saúde; centros de controle de zoonoses; distribuidores de produtos farmacêuticos; importadores, distribuidores e produtores de materiais e controles para diagnóstico in vitro; unidades móveis de atendimento à saúde; serviços de acupuntura; serviços de tatuagem, entre outros similares (CONAMA, 2005b).

	A Resolução do Conselho Nacional do Meio Ambiente (CONAMA) 358/2005 classifica ainda os resíduos de serviços de saúde de acordo com os riscos potenciais ao meio ambiente e à saúde pública, conforme mostrado na tabela 16 (CONAMA, 2005a).
	
	A tabela 17 apresenta o objetivo de tratamento dos respectivos grupos de RSS para que estes resíduos possam ser destinados adequadamente de acordo com o determinado pela Resolução CONAMA n$\deg$ 358/2005 (CONAMA, 2005a).
	
	A adequada segregação, acondicionamento e armazenamento dos RSS gerados é de suma importância para a redução de riscos à saúde que o trabalhador e usuário está exposto, à população e ao meio ambiente (ANVISA, 2007; BRASIL, 2005b).
	
	Os resíduos gerados nesse segmento são de responsabilidade dos seus geradores, devendo os mesmos, portanto, realizar o correto gerenciamento atendendo às normas e exigências legais, desde a geração até sua disposição final, de acordo com a Resolução RDC nº 306/2004 (BRASIL, 2004). Esta resolução ainda estabelece que os estabelecimentos geradores são responsáveis pela elaboração do Plano de Gerenciamento de Resíduos de Serviços de Saúde – PGRSS, o qual deve contemplar e descrever as etapas de geração, segregação, acondicionamento, coleta interna, armazenamento, coleta externa, transporte, tratamento e disposição final (BRASIL, 2004).
	
	De acordo com o artigo 4$\deg$ da Resolução CONAMA 358/2005, compete aos órgãos ambientais do Município, do Estado e do Distrito Federal a determinação de critérios que definam quais estabelecimentos serão passíveis de licenciamento ambiental e de elaborar e implementar o PGRSS (CONAMA, 2005b).
	
	De acordo com o artigo 2$\deg$ e 3$\deg$ da Resolução RDC n$\deg$ 306/2004, compete à Vigilância Sanitária do Município, do Estado e do Distrito Federal promover a divulgação, orientação e fiscalização das condições estabelecidas por esta resolução, além de poderem estabelecer normas complementares para possíveis adequações às especificidades locais (BRASIL, 2004).
	Em relação à geração de RSS no Brasil, a figura 50 compara a quantidade gerada em tonelada para os anos de 2015 e 2016, podendo ser possível verificar uma discreta redução na geração entre os anos (ABRELPE, 2016).
	
	Os resíduos de serviços de saúde apresentam um leque variado de possibilidades de tratamento e destinação final (ABRELPE, 2016). Desse modo, a figura 51 apresenta os tipos de destinação final de RSS realizadas pelos municípios brasileiros.
	
	Observando o gráfico nota-se que aproximadamente 26 \% dos resíduos são enquadrados como outros tipos de destinação. Esta classificação refere-se às destinações que não apresentam tratamento prévio, ou seja, aterros sanitários, valas sépticas, lixões (ABRELPE, 2016). Esta realidade expõe a necessidade de elaboração e implementação de PMGIRS e PGRSS para que estes resíduos apresentem uma disposição final ambientalmente adequada e segura para o meio ambiente.
	Do total de RSS gerado no ano, entre 10 e 25 \% do volume total foi classificado como um resíduo do grupo A, B, C ou E em 2011. Dessa forma, a maioria dos RSS coletados correspondem a resíduos do grupo D, que são resíduos comuns e passíveis de reciclagem (BRASIL, 2011a).
	
	No estado de São Paulo, a produção em 2016 atingiu uma quantidade de 161.643 toneladas de RSS, sendo equivalente a 2,271 kg/hab./ano (ABRELPE, 2016).
	
	O município de Monteiro Lobato dispõe de uma Unidade Básica de Saúde (UBS) denominada de Centro de Saúde “Dr. João Auricchio”, é uma unidade mista que realiza serviço 24 horas de pronto atendimento e atendimento de ambulatório de segunda-feira à sexta-feira. Além disso, de acordo com disponibilizado pelos secretariados em resposta à ofício, o município é composto por sete principais estabelecimentos geradores de RSS que englobam veterinários, consultórios odontológicos, farmácia e estabelecimento de comércio de produtos para animais. 
	
	Todos estes estabelecimentos geradores de RSS foram visitados e responderam à um questionário estruturado. O questionário estruturado apresentava perguntas sobre quais tipos de RSS gerados, de acordo com as classificações determinadas pela Resolução CONAMA 358/2005, se havia a segregação entre resíduo sólido comum e resíduo sólido reciclável, quais as respectivas formas de acondicionamento, qual a destinação final e se o estabelecimento apresentava PGRSS.
	
	Tanto a UBS quanto a maioria dos estabelecimentos geradores de 
    RSS não dispõem de PGRSS. Apesar disso, um dos estabelecimentos de veterinária e a farmácia apresentam um Manual que apresenta práticas e recomendações para o adequado gerenciamento dos RSS. Assim, a elaboração do PGRSS para estes dois geradores particulares de RSS poderá menos dificultosa.
	
	De todos os estabelecimento público e particulares geradores de RSS, apenas um dos estabelecimentos veterinários não realiza a segregação entre resíduos sólidos comuns e resíduos sólidos reciclável.
	
	O resíduo do estabelecimento de comércio de produtos para animais que se enquadra como RSS se trata de vacinas e medicamentos que são vendidos para tratamento de animais. De acordo com o dono do estabelecimento, estes resíduos se tornam responsabilidade do comprador quando vendidos e, no caso de atingirem a data de vencimento antes de serem vendidos, o fornecedor realiza o recolhimento.
	
	\subsubsection{Acondicionamento}
	A tabela 14 apresenta a relação de geração de resíduos sólidos e condição de acondicionamento. A forma de acondicionamento foi classificada como adequada ou inadequada de acordo com o recomendado pelo Manual de Gerenciamento de RSS orientado pela Agência Nacional de Vigilância Sanitária (ANVISA) (BRASIL, 2006). 
	
	A ANVISA recomenda que os RSS classificados como grupo A sejam acondicionados em sacos plásticos branco leitoso ou vermelho identificados com a simbologia da substância infectante. Estes devem ser contidos em recipientes resistentes à tombamentos, que respeitem aos limites de peso de cada invólucro, sejam laváveis, resistentes à punctura, ruptura e vazamentos e apresente tampas de abertura sem contato manual (BRASIL, 2006).
	
	Os RSS de classe B, quando se tratando de substâncias perigosas, devem ser acondicionados e descartadas de acordo com as recomendações estabelecidas pelo fabricante. Os resíduos sólidos devem ser acondicionados em recipientes de material rígido e específico para cada tipo de substância química. Já os resíduos líquidos devem ser acondicionados em recipientes de material compatível com o resíduo, além de ser resistente, rígido, com tampa rosqueada e vedante. Tanto os resíduos sólidos quanto líquidos devem ser contidos em recipientes com as respectivas identificações (BRASIL, 2006).
	
	Os rejeitos radioativos classificados do Grupo C devem ser acondicionados em recipientes de chumbo apresentando a blindagem necessária e simbologia específica. Os rejeitos radioativos sólidos devem ser acondicionados em sacos plásticos resistentes e com identificação e contidos em recipientes de material rígido. Enquanto que os rejeitos radioativos líquidos devem ser acondicionados em frascos ou bombonas de material compatível com o resíduo que apresentem resistência, rigidez, tampa de rosca e vedação (BRASIL, 2006).
	
	Resíduos do Grupo D devem ser acondicionados em sacos plásticos impermeáveis. Os cadáveres de animais devem apresentar acondicionamento e transporte específicos para seu porte e atendendo ao exigido e aprovado pelo órgão de limpeza urbana responsável pela gestão dos resíduos sólidos urbanos (BRASIL, 2006).
	
	Os resíduos cortantes e perfurocortantes do Grupo E devem ser acondicionados em recipiente que apresente rigidez, estanque, resistência à punctura, ruptura e vazamento, tampa e simbologia da substância (BRASIL, 2006).
	
	A classificação de “Adequado” na tabela 18 significa que a forma de acondicionamento dos resíduos encontra-se conforme a recomendação do manual de gerenciamento. A partir do não atendimento de duas ou mais recomendações, o acondicionamento foi classificado como “Inadequado”. A coluna de observação descreve qual recomendação não é atendida.
	
	De acordo com as informações apresentadas na tabela 14, verifica-se que os estabelecimentos não geram resíduos radioativos e que são poucos os que geram resíduos compostos por substâncias químicas. As chapas de raio-X são os principais resíduos considerados de classe B. A UBS atualmente utiliza equipamento digital para análise da chapa de raio X, porém ainda recebe eventualmente chapas de pacientes. Estas chapas são guardadas em ambiente fechado e não dispõem ainda de um descarte específico.
	
	O consultório odontológico 1 guarda todas as chapas em ambiente fechado e adequado, enquanto que o consultório odontológico 2 costuma encaminhar as chapas para o município de São José dos Campos para descarte.
	Para o acondicionamento dos resíduos de cortantes e perfurocortantes, todos os estabelecimentos geradores utilizam o “Descarpack” apresentado na figura 52 que é disponibilizado pela UBS.
	
	Já os resíduos contaminados como materiais com sangue, luvas e gases são descartados por quase todos os estabelecimentos em sacos brancos leitosos. No entanto, nem todos apresentam a identificação no saco, como recomendado. O acondicionamento destes materiais contaminados da UBS é adequado e é apresentado na figura 53.
	
	Os resíduos similares aos domiciliares, como dito anteriormente, são segregados entre comum e reciclável pela maioria dos estabelecimentos geradores de RSS. A UBS segrega estes resíduos e alguns recipientes são envolvidos com sacos plásticos pretos, como mostrado na figura 54. No entanto, alguns dos recipientes não apresentam não são envolvidos com nenhum saco plástico, como apresentado na figura 55.
	
	Além disso, a cozinha da UBS não dispõe de recipiente para descarte de resíduos orgânicos e recicláveis, o que faz com que seja descartado no recipiente disponível resíduos recicláveis (copos), orgânicos (cascas de frutas) e rejeitos (guardanapos usados) como revelado na figura 56. 
	
	A UBS dispõe de local de armazenamento externo, apresentado na figura 57, onde os resíduos gerados são armazenados até que os resíduos similares aos resíduos sólidos domiciliares (armazenados na primeira porta) sejam dispostos no muro para coleta ou até que os resíduos dos grupos A e E (armazenados na segunda porta) sejam coletados pela empresa terceirizada especializada.
	
	De acordo com o recomendado pelo Manual de Gerenciamento de RSS da ANVISA, o local de armazenamento externo apresenta área suficiente para armazenamento dos resíduos, ambiente separado para armazenamento de resíduos do grupo D e do grupo A em conjunto com o Grupo E e condições para lavagem adequadas (pisos e paredes laváveis, ralos). No entanto, os sacos deveriam ser armazenados em recipientes rígidos ao invés de serem dispostos diretamente no piso, como apresentado na figura 58 (BRASIL. 2006).
	
	Os resíduos compostos pelos grupos A e E são coletados pela empresa terceirizada AGIT Soluções Ambientais Ltda diretamente do local de armazenamento temporário.
	
	Os sacos plásticos contendo os resíduos sólidos comum e recicláveis são atualmente dispostos no muro da UBS até que o serviço municipal de coleta seja realizado, como apresentado na figura 59.
	
	Para a segurança e higiene, seria recomendado que estes sacos fossem armazenados em cestas ou locais fechados até o momento da coleta. Os resíduos sólidos comuns e recicláveis são dispostos nos dias e horários das respectivas coletas, no entanto, devido ao horário restrito da coleta de resíduos recicláveis (terças e quintas no período da tarde), seria preferível que houvesse também cestas ou locais fechados específicos para cada um dos resíduos comum e reciclável.
	De acordo com o recomendado pelo Manual de gerenciamento da ANVISA, o local de armazenamento externo deveria apresentar (BRASIL, 2006):
	
	\begin{description}
		\item [Acessibilidade] Ambiente que permita fácil acesso para recipientes de transporte e para os veículos coletores.
		\item[Exclusividade] Ambiente utilizado somente para armazenamento de resíduos.
		\item[Segurança] Ambiente deve ser seguro da ação de intemperes e do contato de animais e pessoas não autorizadas.
		Higiene e saneamento: Deve haver local para limpeza dos recipientes e contenedores, o ambiente deve apresentar iluminação, ventilação e pisos e paredes laváveis
	\end{description}

	\subsubsection{Coleta, transbordo e transporte} 
	Os estabelecimentos particulares geradores de RSS encaminham os resíduos classificados como grupo A e E para a Unidade Básica de Saúde de Monteiro Lobato, com exceção do veterinário 2, que encaminha estes resíduos para São José do Campos. Os resíduos do Grupo D gerados pelos estabelecimentos público e particulares são coletados pelo serviço de coleta público do município.
	
	A UBS armazena no ambiente específico os resíduos de grupo A e E gerados e recolhidos e, de 15 em 15 dias a empresa AGIT Soluções Ambientais Ltda é responsável pela coleta, transporte para o município de Itajubá e disposição final destes resíduos.
	
	De acordo com os dados adquiridos da Série histórica do SNIS, os municípios de Lagoinha e São José do Barreiro também executam o serviço de coleta diferenciada por empresas contratadas e encaminha os resíduos para outros municípios, de modo que Lagoinha e São José do Barreiro declararam até 2015 encaminhar os resíduos para Guaratinguetá e São Bernardo do Campo, respectivamente (SNIS, 2009 a 2015). 
	
	Através de entrevistas com os responsáveis pelos estabelecimentos veterinário, os donos são responsáveis pela disposição final do corpo do animal quando o mesmo vem à óbito quando tratado no veterinário 1, enquanto que o estabelecimento veterinário 2 ou costuma enterrar o cadáver do animal.
	
	1.7.3. Disposição final
	De acordo com o contrato realizado entre a Prefeitura e Monteiro Lobato e a empresa especializada AGIT Soluções Ambientais Ltda, a empresa é responsável pelo serviço de coleta, transporte e incineração dos resíduos de serviço de saúde de até 1.800 quilogramas coletados da UBS.
	O contrato com a empresa especializada foi firmado em setembro de 2014 e têm sido anualmente prorrogado para os mesmos serviços por mais 12 meses até atualmente, no ano de 2017.