% Table generated by Excel2LaTeX from sheet 'tratamento_rss'
\begin{table}[htbp]
  \centering
  \caption{ Opções de tratamento e respectivas destinações finais para cada grupo de RSS.}
    \begin{tabular}{c|p{12.07em}|p{20.57em}}
    \rowcolor[rgb]{ .929,  .49,  .192} \multicolumn{1}{c}{\textcolor[rgb]{ 1,  1,  1}{}} & \multicolumn{1}{c}{\textcolor[rgb]{ 1,  1,  1}{\textbf{Tratamento}}} & \multicolumn{1}{c}{\textcolor[rgb]{ 1,  1,  1}{\textbf{Destinação Final}}} \\
    \rowcolor[rgb]{ .957,  .69,  .514} \textbf{Grupo A} & Tratamento que reduza a carga microbiana. & Aterro sanitário licenciado ou local devidamente licenciado para disposição final de RSS. Peças anatômicas devem ser incineradas ou cremadas. Em casos específicos, sepultamento em cemitério \\
    \rowcolor[rgb]{ .984,  .894,  .835} \textbf{Grupo B} & Tratamento, reutilização, reciclagem de acordo com periculosidade. & Quando não tratados, devem ser dispostos em aterro de resíduos perigosos classe 1. Quando não perigosos, podem ser dispostos em aterro licenciado. \\
    \rowcolor[rgb]{ .957,  .69,  .514} \textbf{Grupo C} & Tratamento obedecendo condições de CNEN & Destinação final de acordo com exigência de CNEN \\
    \rowcolor[rgb]{ .992,  .914,  .851} \textbf{Grupo D} & Reutilização, Recuperação, Reciclagem quando possível & Dispostos em aterro sanitário de RSU quando não passíveis de reutilização, recuperação e reciclagem \\
    \rowcolor[rgb]{ .957,  .69,  .514} \textbf{Grupo E} & Tratamento específico de acordo com a contaminação química, biológica ou radiológica & Disposição final de acordo com contaminação e tratamento. \\
    \end{tabular}%
  \label{tab:tratamento_rss}%
  \legend{Fonte: Adaptado de CONAMA, 2005b.}
\end{table}%
