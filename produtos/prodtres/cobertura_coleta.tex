% Table generated by Excel2LaTeX from sheet 'cobertura_coleta'
\begin{table}[htbp]
  \centering
  \caption{Porcentagem de cobertura de coleta de RSU em área urbana e rural no Brasil e nas respectivas regiões.}
  \arrayrulecolor{white}
    \begin{tabular}{|c|c|c|c|c|c|c|c|c|c|}
\cmidrule{4-10}    \rowcolor[rgb]{ .969,  .588,  .275} \textcolor[rgb]{ 1,  1,  1}{} & \textcolor[rgb]{ 1,  1,  1}{\textbf{2000}} & \textcolor[rgb]{ 1,  1,  1}{\textbf{2001}} & \textcolor[rgb]{ 1,  1,  1}{\textbf{2002}} & \textcolor[rgb]{ 1,  1,  1}{\textbf{2003}} & \textcolor[rgb]{ 1,  1,  1}{\textbf{2004}} & \textcolor[rgb]{ 1,  1,  1}{\textbf{2005}} & \textcolor[rgb]{ 1,  1,  1}{\textbf{2006}} & \textcolor[rgb]{ 1,  1,  1}{\textbf{2007}} & \textcolor[rgb]{ 1,  1,  1}{\textbf{2008}} \\
    \rowcolor[rgb]{ .984,  .831,  .706} \textbf{BRASIL (\%)} & 83,2  & 84,8  & 85,6  & 84,7  & 85,7  & 86,5  & 87,3  & 87,9  & 88,6 \\
    \textbf{Urbano (\%)} & 94,9  & 95,9  & 96,5  & 96,3  & 97    & 97,4  & 97,9  & 98,1  & 98,5 \\
    \rowcolor[rgb]{ .992,  .914,  .851} \textbf{Rural (\%)} & 15,7  & 18,6  & 20,5  & 21,6  & 23,9  & 26    & 28,4  & 30,2  & 32,7 \\
    \rowcolor[rgb]{ .984,  .831,  .706} \textbf{NORTE (\%)} & 82,2  & 85,1  & 85,7  & 71,3  & 74,1  & 76,6  & 79    & 80,1  & 82,2 \\
    \rowcolor[rgb]{ .992,  .914,  .851} \textbf{Urbano (\%)} & 85,3  & 88,1  & 88,6  & 88,9  & 91,6  & 93,5  & 95,2  & 95,7  & 97,1 \\
    \textbf{Rural (\%)} & -     & -     & -     & 17    & 19,2  & 20,6  & 23,3  & 24,9  & 29,4 \\
    \rowcolor[rgb]{ .984,  .831,  .706} \textbf{NORDESTE  (\%)} & 66,3  & 68,5  & 70,1  & 69,8  & 71,9  & 72,8  & 73,9  & 75,4  & 76,2 \\
    \textbf{Urbano (\%)} & 88,4  & 90,3  & 91,8  & 90,8  & 92,8  & 93,3  & 94,3  & 95,3  & 95,8 \\
    \rowcolor[rgb]{ .992,  .914,  .851} \textbf{Rural (\%)} & 8,7   & 10,2  & 11,6  & 11,4  & 15    & 15,4  & 16,9  & 18,4  & 19,8 \\
    \rowcolor[rgb]{ .984,  .831,  .706} \textbf{SUDESTE (\%)} & 92,3  & 93,6  & 93,9  & 94,2  & 94,4  & 94,8  & 95,3  & 95,3  & 95,9 \\
    \rowcolor[rgb]{ .992,  .914,  .851} \textbf{Urbano (\%)} & 97,8  & 98,5  & 98,6  & 98,7  & 98,9  & 99,1  & 99,3  & 99,2  & 99,5 \\
    \textbf{Rural (\%)} & 27,9  & 34,1  & 35    & 38    & 39    & 42,1  & 44,7  & 47    & 50,5 \\
    \rowcolor[rgb]{ .984,  .831,  .706} \textbf{SUL (\%)} & 84,4  & 85,4  & 86,7  & 87,3  & 87,9  & 89,3  & 90,5  & 90,7  & 91,5 \\
    \textbf{Urbano (\%)} & 98,1  & 98,4  & 98,7  & 98,8  & 98,8  & 99,2  & 99,4  & 99,4  & 99,6 \\
    \rowcolor[rgb]{ .992,  .914,  .851} \textbf{Rural (\%)} & 20,6  & 23,6  & 28,2  & 30,7  & 32,5  & 38,8  & 44,2  & 46,2  & 49 \\
    \rowcolor[rgb]{ .984,  .831,  .706} \textbf{CENTRO-OESTE (\%)} & 84,4  & 85,8  & 86,1  & 86,7  & 87,1  & 87,8  & 88,2  & 89,2  & 89,9 \\
    \rowcolor[rgb]{ .992,  .914,  .851} \textbf{Urbano (\%)} & 95,7  & 96,7  & 97,5  & 97,4  & 98,1  & 98,7  & 98,6  & 98,9  & 98,8 \\
    \textbf{Rural (\%)} & 11,4  & 13,5  & 15,4  & 20,4  & 19,6  & 19,5  & 21,7  & 21,8  & 26,4 \\
\cmidrule{4-10}    
\end{tabular}%
  \label{tab:cobertura_coleta}%
  \legend{Fonte: Adaptado de IPEA, 2012a.}
\end{table}%
