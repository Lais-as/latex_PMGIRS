\begin{table}[htbp]
\caption{Tipos de RSU e suas respectivas composições utilizados na caracterização de Monteiro Lobato de 2017.}
\begin{center}
\begin{tabular}{|l|l|}
\hline
\multicolumn{1}{|c|}{\textbf{Tipo de RSU}} & \multicolumn{1}{c|}{\textbf{Composição}} \\ \hline
\textit{Materiais Orgânicos} & compostos por restos de alimentos processados (arroz, feijão, carne, etc.), frutas e legumes, animais mortos e restos de poda e capina \\ \hline
\textit{Materiais Higiênicos} & compostos por fraldas descartáveis, papeis de higiene pessoal e quaisquer outros materiais que foram contaminados por estarem acondicionados no mesmo recipiente contendo os materiais higiênicos \\ \hline
\textit{Tecidos} & compostos por trapos, panos para costura, roupas, tapetes não emborrachados e carpetes, brinquedos (bonecos de pano) \\ \hline
\textit{Plásticos Finos} & compostos por sacos plásticos de lixo, de mercado, de presente, de proteção de eletrônicos, plásticos filmes, embalagens de proteção para alimento e isopor \\ \hline
\textit{Plásticos Duros} & compostos por garrafas PET, galões de produtos de limpeza, baldes, tapetes de borracha, mangueiras, brinquedos de plástico (carrinhos, bonecos, miniaturas de casas e cozinhas) \\ \hline
\textit{Papel/Papelão} & Compostos por revistas, jornais, cadernos, apostilas, panfletos, fotos, caixas de papelão, tubos de papelão
 \\ \hline
\textit{Metais Mistos} & compostos por metais ferrosos e não ferrosos (ferro, zinco, cobre, alumínio, aço entre outras ligas) em formatos de fios, barras, embalagens (latinhas), chapas e instrumentos (facas, tesouras, serras etc.) \\ \hline
\textit{Vidros} & compostos por garrafas, embalagens de perfume, espelhos, copos, porcelanas \\ \hline
\textit{Madeiras} & compostas por recorte de madeiras processadas, moveis, caibro, tacos para piso, artesanatos em madeira \\ \hline
\textit{Pneus} & compostos por pneus de pequenos e grandes veículos (motos, carros, caminhões) e pneus de bicicletas \\ \hline
\textit{Outros} & compostos por materiais da logística reversa (embalagens de óleos automotivos, lâmpadas fluorescentes, equipamentos eletroeletrônicos e pilhas) e, materiais de RSS (seringas, medicamentos e embalagens medicamentosas) \\ \hline
\end{tabular}
\end{center}
\label{tab:tipos_rsu}
\end{table}
