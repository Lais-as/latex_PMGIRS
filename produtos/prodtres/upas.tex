% Table generated by Excel2LaTeX from sheet 'upas'
\begin{table}[htbp]
  \centering
  \caption{UPAs contabilizadas no município de Monteiro Lobato.}
    \begin{tabular}{c|c|c|c|c|c}
    \rowcolor[rgb]{ .969,  .588,  .275} \textcolor[rgb]{ 1,  1,  1}{\textbf{Área}} & \multicolumn{1}{p{4.215em}|}{\textcolor[rgb]{ 1,  1,  1}{\textbf{N° de UPA}}} & \multicolumn{1}{p{4.215em}|}{\textcolor[rgb]{ 1,  1,  1}{\textbf{Área mínima (ha)}}} & \multicolumn{1}{p{4.215em}|}{\textcolor[rgb]{ 1,  1,  1}{\textbf{Área média (ha)}}} & \multicolumn{1}{p{4.215em}|}{\textcolor[rgb]{ 1,  1,  1}{\textbf{Área máxima (ha)}}} & \multicolumn{1}{p{4.215em}}{\textcolor[rgb]{ 1,  1,  1}{\textbf{TOTAL}}} \\
    \rowcolor[rgb]{ .992,  .914,  .851} \textbf{Área Total} & 314   & 2     & 83,3  & 5200  & 26162,8 \\
    \rowcolor[rgb]{ .984,  .831,  .706} \textbf{Área com cultura perene} & 45    & 0,3   & 1,8   & 7,2   & 79,3 \\
    \rowcolor[rgb]{ .992,  .914,  .851} \textbf{Área com cultura temporária} & 32    & 0,2   & 2,6   & 23,4  & 84,1 \\
    \rowcolor[rgb]{ .984,  .831,  .706} \textbf{Área com pastagens} & 283   & 0,5   & 50,6  & 3000  & 14325,6 \\
    \rowcolor[rgb]{ .992,  .914,  .851} \textbf{Área com reflorestamento} & 59    & 0,3   & 57,2  & 1000  & 3376,5 \\
    \rowcolor[rgb]{ .984,  .831,  .706} \textbf{Área com vegetação natural} & 275   & 0,2   & 23,1  & 2000  & 6357,4 \\
    \rowcolor[rgb]{ .992,  .914,  .851} \textbf{Área com vegetação de brejo e várzea} & 31    & 0,1   & 2,2   & 24,8  & 69,5 \\
    \rowcolor[rgb]{ .984,  .831,  .706} \textbf{Área em descanso} & 35    & 1,5   & 39,9  & 280,4 & 1397,9 \\
    \rowcolor[rgb]{ .992,  .914,  .851} \textbf{Área complementar} & 308   & 0,1   & 1,5   & 100   & 472,5 \\
    \end{tabular}%
  \label{tab:upas}%
  \legend{Fonte: Adaptado de ESTADO DE SÃO PAULO, 2007/2008.}
\end{table}%
