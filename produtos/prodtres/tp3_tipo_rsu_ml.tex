% Table generated by Excel2LaTeX from sheet 'Planilha1'
\begin{table}[htbp]
  \centering
  \caption{Tipos de RSU e suas respectivas composições utilizados na caracterização de Monteiro Lobato de 2017.}
    \begin{tabular}{p{11.07em}p{26.93em}}
    \rowcolor[rgb]{ .969,  .588,  .275} \textbf{Tipo de RSU} & \textbf{Composição} \\
    \rowcolor[rgb]{ .992,  .914,  .851} \textit{Materiais Orgânicos} & compostos por restos de alimentos processados (arroz, feijão, carne, etc.), frutas e legumes, animais mortos e restos de poda e capina \\
    \rowcolor[rgb]{ .984,  .831,  .706} \textit{Materiais Higiênicos} & compostos por fraldas descartáveis, papeis de higiene pessoal e quaisquer outros materiais que foram contaminados por estarem acondicionados no mesmo recipiente contendo os materiais higiênicos \\
    \rowcolor[rgb]{ .992,  .914,  .851} \textit{Tecidos} & compostos por trapos, panos para costura, roupas, tapetes não emborrachados e carpetes, brinquedos (bonecos de pano) \\
    \rowcolor[rgb]{ .984,  .831,  .706} \textit{Plásticos Finos} & compostos por sacos plásticos de lixo, de mercado, de presente, de proteção de eletrônicos, plásticos filmes, embalagens de proteção para alimento e isopor \\
    \rowcolor[rgb]{ .992,  .914,  .851} \textit{Plásticos Finos} & compostos por garrafas PET, galões de produtos de limpeza, baldes, tapetes de borracha, mangueiras, brinquedos de plástico (carrinhos, bonecos, miniaturas de casas e cozinhas) \\
    \rowcolor[rgb]{ .984,  .831,  .706} \textit{Papel/Papelão} & Compostos por revistas, jornais, cadernos, apostilas, panfletos, fotos, caixas de papelão, tubos de papelão \\
    \rowcolor[rgb]{ .992,  .914,  .851} \textit{Metais Mistos} & compostos por metais ferrosos e não ferrosos (ferro, zinco, cobre, alumínio, aço entre outras ligas) em formatos de fios, barras, embalagens (latinhas), chapas e instrumentos (facas, tesouras, serras etc.) \\
    \rowcolor[rgb]{ .984,  .831,  .706} \textit{Tetra Pak} & compostos por caixas de leite, sucos, achocolatados e doces \\
    \rowcolor[rgb]{ .992,  .914,  .851} \textit{Isopor} & composto por poça ou caixa de isopor e estrutura de isopor contra impacto de eletroeletrônicos \\
    \rowcolor[rgb]{ .984,  .831,  .706} \textit{Outros} & compostos por materiais da logística reversa (embalagens de óleos automotivos, lâmpadas fluorescentes, equipamentos eletroeletrônicos e pilhas) e, materiais de RSS (seringas, medicamentos e embalagens medicamentosas) \\
    \end{tabular}%
  \label{tab:tipo_rsu_ml}%
\end{table}%
