% Table generated by Excel2LaTeX from sheet 'manejo_rcc'
\begin{table}[htbp]
  \centering
  \caption{Informação sobre o tipo de processamento entre os 392 municípios brasileiros com serviço de manejo de RCC.}
    \begin{tabular}{P{28.07em}|c}
    \rowcolor[rgb]{ .969,  .588,  .275} \textcolor[rgb]{ 1,  1,  1}{\textbf{Tipos de Processamento}} & \multicolumn{1}{P{6.645em}}{\textcolor[rgb]{ 1,  1,  1}{\textbf{Quantidade}}} \\
    \rowcolor[rgb]{ .992,  .914,  .851} Reaproveitamento dos agregados produzidos na fabricação de componentes construtivos & 79 \\
    \rowcolor[rgb]{ .984,  .831,  .706} Triagem e trituração simples dos resíduos Classe A, com classificação granulométrica dos agregados reciclados & 20 \\
    \rowcolor[rgb]{ .992,  .914,  .851} Triagem e trituração simples dos resíduos Classe A & 14 \\
    \rowcolor[rgb]{ .984,  .831,  .706} Triagem simples dos resíduos de construção e demolição reaproveitáveis (classes A e B) & 124 \\
    \rowcolor[rgb]{ .992,  .914,  .851} Outros & 204 \\
    \end{tabular}%
  \label{tab:manejo_rcc}%
  \legend{Fonte: Adaptado de (INSTITUTO DE PESQUISA ECONÔMICA APLICADA, 2012).}
\end{table}%
