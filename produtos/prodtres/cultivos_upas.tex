% Table generated by Excel2LaTeX from sheet 'cultivos_upa'
\begin{table}[htbp]
  \centering
  \caption{Cultivos Realizados nas UPAs de Monteiro Lobato.}
    \begin{tabular}{p{14.5em}|c|c|c|c|c}
    \rowcolor[rgb]{ .969,  .588,  .275} \textcolor[rgb]{ 1,  1,  1}{\textbf{CULTURA}} & \multicolumn{1}{P{4.215em}|}{\textcolor[rgb]{ 1,  1,  1}{\textbf{N. de UPA}}} & \multicolumn{1}{P{4.215em}|}{\textcolor[rgb]{ 1,  1,  1}{\textbf{Área mínima (ha)}}} & \multicolumn{1}{P{4.215em}|}{\textcolor[rgb]{ 1,  1,  1}{\textbf{Área média (ha)}}} & \multicolumn{1}{P{5.07em}|}{\textcolor[rgb]{ 1,  1,  1}{\textbf{Área máxima (ha)}}} & \multicolumn{1}{P{5.43em}}{\textcolor[rgb]{ 1,  1,  1}{\textbf{TOTAL}}} \\
    \rowcolor[rgb]{ .984,  .831,  .706} \textbf{Braquiária} & 254   & 1     & 43,7  & 1.500,00 & 11.110,20 \\
    \rowcolor[rgb]{ .992,  .914,  .851} \textbf{Eucalipto} & 59    & 0,3   & 57,2  & 1.000,00 & 3.376,50 \\
    \rowcolor[rgb]{ .984,  .831,  .706} \textbf{Outras gramíneas para pastagem} & 30    & 0,3   & 79,5  & 1.500,00 & 2.386,40 \\
    \rowcolor[rgb]{ .992,  .914,  .851} \textbf{Capim-gordura} & 18    & 3     & 25,7  & 72,8  & 462,9 \\
    \rowcolor[rgb]{ .984,  .831,  .706} \textbf{Gramas} & 8     & 2,3   & 39,3  & 191   & 314,3 \\
    \rowcolor[rgb]{ .992,  .914,  .851} \textbf{Capim-napier (capim-elefante)} & 30    & 0,4   & 1,7   & 5     & 50,8 \\
    \rowcolor[rgb]{ .984,  .831,  .706} \textbf{Banana} & 21    & 0,4   & 1,8   & 7,2   & 38,1 \\
    \rowcolor[rgb]{ .992,  .914,  .851} \textbf{Cana-de-açúcar} & 17    & 0,3   & 1,9   & 5     & 32,3 \\
    \rowcolor[rgb]{ .984,  .831,  .706} \textbf{Outras culturas temporárias} & 1     & 23,4  & 23,4  & 23,4  & 23,4 \\
    \rowcolor[rgb]{ .992,  .914,  .851} \textbf{Milho} & 13    & 0,2   & 1,3   & 3     & 17,4 \\
    \rowcolor[rgb]{ .984,  .831,  .706} \textbf{Outras frutíferas} & 10    & 0,5   & 1,5   & 4     & 14,9 \\
    \rowcolor[rgb]{ .992,  .914,  .851} \textbf{Café} & 7     & 0,5   & 2,1   & 6     & 14,5 \\
    \rowcolor[rgb]{ .984,  .831,  .706} \textbf{Pomar doméstico} & 11    & 0,5   & 0,6   & 1     & 7,1 \\
    \rowcolor[rgb]{ .992,  .914,  .851} \textbf{Mandioca} & 5     & 0,5   & 1,1   & 2     & 5,5 \\
    \rowcolor[rgb]{ .984,  .831,  .706} \textbf{Laranja} & 3     & 1     & 1,1   & 1,2   & 3,2 \\
    \rowcolor[rgb]{ .992,  .914,  .851} \textbf{Feijão} & 2     & 1     & 1,3   & 1,5   & 2,5 \\
    \rowcolor[rgb]{ .984,  .831,  .706} \textbf{Milho-silagem} & 1     & 2     & 2     & 2     & 2 \\
    \rowcolor[rgb]{ .992,  .914,  .851} \textbf{Limão} & 1     & 1,2   & 1,2   & 1,2   & 1,2 \\
    \rowcolor[rgb]{ .984,  .831,  .706} \textbf{Colonião} & 1     & 1     & 1     & 1     & 1 \\
    \rowcolor[rgb]{ .992,  .914,  .851} \textbf{Outras flores} & 1     & 1     & 1     & 1     & 1 \\
    \rowcolor[rgb]{ .984,  .831,  .706} \textbf{Jabuticaba} & 1     & 0,3   & 0,3   & 0,3   & 0,3 \\
    \end{tabular}%
  \label{tab:cultivos_upas}%
  \legend{Fonte: Adaptado de ESTADO DE SÃO PAULO, 2007/2008.}
\end{table}%
