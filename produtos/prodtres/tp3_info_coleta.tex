\begin{table}[htbp]
\caption{Informações de coleta e geração de Resíduos Recicláveis no Brasil.}
\begin{center}
\begin{tabular}{|c|c|c|c|c|}
\hline
\textbf{Região} & \textbf{Taxa de cobertura porta-a-porta em relação à população urbana (\%)} & \textbf{Taxa de material recolhido em relação à quantidade total coletada de resíduos sól. Domésticos * (\%)} & \textbf{Massa per capita de materiais recicláveis recolhidos via coleta seletiva (Kg/hab./ano)} & \textbf{Massa recuperada per capita de materiais recicláveis em relação à população urbana (Kg/hab./ano)} \\ \hline
\textbf{Brasil} & \textbf{50,92} & \textbf{6,05} & \textbf{19} & \textbf{9,39} \\ \hline
\textbf{Centro-Oeste} & 55 & 7,18 & 22 & 10,88 \\ \hline
\textbf{Nordeste} & 13,68 & 2,72 & 9 & 5,02 \\ \hline
\textbf{Norte} & 15,1 & 2,41 & 8 & 12,37 \\ \hline
\textbf{Sudeste} & 54,31 & 3,43 & 13 & 6,17 \\ \hline
\textbf{Sul} & 77,3 & 20,38 & 49 & 21,8 \\ \hline
\multicolumn{ 5}{|l|}{\textbf{* exceto matéria orgânica e rejeitos}} \\ \hline
\end{tabular}
\end{center}
\label{tab:info_coleta}
Fonte: Adaptado de série histórica SNIS, 2009 a 2015.
\end{table}
