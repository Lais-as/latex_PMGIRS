\begin{table}[htbp]
\begin{center}
	\caption{Informações de coleta e geração de Resíduos Recicláveis no Brasil.}
	\begin{tabular}{ccccc}
		\rowcolor[rgb]{ .969,  .588,  .275} \multicolumn{1}{p{6.355em}}{\textcolor[rgb]{ 1,  1,  1}{\textbf{Região}}} & \multicolumn{1}{p{9.855em}}{\textcolor[rgb]{ 1,  1,  1}{\textbf{Taxa de cobertura porta-a-porta em relação à população urbana (\%)}}} & \multicolumn{1}{p{9.855em}}{\textcolor[rgb]{ 1,  1,  1}{\textbf{Taxa de material recolhido em relação à quantidade total coletada de resíduos sól. domésticos * (\%)}}} & \multicolumn{1}{p{9.855em}}{\textcolor[rgb]{ 1,  1,  1}{\textbf{Massa per capita de materiais recicláveis recolhidos via coleta seletiva (Kg/hab./ano)}}} & \multicolumn{1}{p{9.855em}}{\textcolor[rgb]{ 1,  1,  1}{\textbf{Massa recuperada per capita de materiais recicláveis em relação à população urbana (Kg/hab./ano)}}} \\
		\rowcolor[rgb]{ .984,  .831,  .706} \textbf{Brasil} & \textbf{50,92} & \textbf{6,05} & \textbf{19} & \textbf{9,39} \\
		\rowcolor[rgb]{ .992,  .914,  .851} \textbf{Centro-Oeste} & 55    & 7,18  & 22    & 10,88 \\
		\rowcolor[rgb]{ .984,  .831,  .706} \textbf{Nordeste} & 13,68 & 2,72  & 9     & 5,02 \\
		\rowcolor[rgb]{ .992,  .914,  .851} \textbf{Norte} & 15,1  & 2,41  & 8     & 12,37 \\
		\rowcolor[rgb]{ .984,  .831,  .706} \textbf{Sudeste} & 54,31 & 3,43  & 13    & 6,17 \\
		\rowcolor[rgb]{ .992,  .914,  .851} \textbf{Sul} & 77,3  & 20,38 & 49    & 21,8 \\
		\rowcolor[rgb]{ .984,  .831,  .706} \multicolumn{5}{c}{\textbf{* exceto matéria orgânica e rejeitos}} \\
	\end{tabular}%
\label{tab:info_coleta}
Fonte: Adaptado de série histórica SNIS, 2009 a 2015.
\end{center}
\end{table}
