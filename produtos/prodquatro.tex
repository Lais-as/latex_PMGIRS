	\newcolumntype{P}[1]{>{\centering\arraybackslash}p{#1}} %para centralizar as colunas nas tabelas

	\thispagestyle{headfootimage}

	PULL REQUEST lAIS
	
	
	
	
	\subsection{Metodologia}
	A oficina participativa é um instrumento amplamente utilizado para aproximar entidades públicas ou privadas de comunidades que serão diretamente afetadas por ações, empreendimento ou políticas que possam alterar o cotidiano desta população. Nessa prática, procura-se informar as condições dos locais que receberão tais ações propostas, os estudos efetuados e resultados obtidos até o momento. A informação é passada de forma simples, direta e transparente, de forma a cientificar e elucidar o caso à população quanto ao andamento do projeto e às possíveis alterações que ocorrerão dentro do escopo apresentado. 
	
	A participação da população nessa etapa é de suma importância para a construção das ações pretendidas no local que sofrerá com essas mudanças. Nesse momento a população tem a possibilidade de fazer críticas e considerações (construtivas ou destrutivas) sobre essas ações, propor alternativas mais condizentes com e a realidade e as necessidades locais, bem como se informar sobre o andamento do empreendimento.
	Em Monteiro Lobato foram estabelecidas 5 oficinas participativas relevantes a serem realizadas, sendo elas:
	
	\begin{itemize}
		\item Professores da Rede Pública
		\item Representantes do Conselho Municipal de Turismo (COMTUR)
		\item Moradores do Bairro Centro
		\item Moradores do Bairro Souzas
		\item Moradores do Bairro São Benedito
	\end{itemize}
	
	A oficina foi estruturada de modo a repassar aos participantes as condições relacionadas à produção, descarte, transporte e destinação dos resíduos sólidos do município; conseguinte a interação em conjunto dos participantes para identificar os principais problemas do município, baseado na percepção dos moradores de Monteiro Lobato, em relação aos resíduos sólidos que foram atribuídos ao grupo, bem como suas soluções viáveis para mitigar tais problemas encontrados. Assim, a oficina foi estruturada em 3 etapas, com 30 minutos cada: 
	
	
	\paragraph{ETAPA 1 (30 minutos)} Apresentação dos dados obtidos e gerados do município.
	\begin{itemize}
		\item Produto 1: Escopo da legislação federal, estadual e municipal referente à área de resíduos sólidos;
		\item Produto 2: Caracterização do meio físico, biótico e antrópico no município;
		\item Produto 3: Diagnóstico dos resíduos sólidos gerados majoritariamente pela população (RSU, RCC, RSS e Logística reversa);
	\end{itemize}
	 
	\paragraph{ETAPA 2 (30 minutos)} Dinâmica inicial dividida em um grupo de pessoas de forma aleatória e proporcional, sendo cada grupo responsável por uma ou duas classes de resíduo (dependendo do número de pessoas presente em cada oficina).
	\begin{itemize}
		\item (15 minutos) Discussão interna sobre problemas relacionados às classes de resíduos apresentadas e os possíveis motivos de ocorrerem, descrevendo-os em postits separados, intitulados como ‘PROBLEMAS’ e ‘POR QUÊ?’;
		\item (15 minutos) Compartilhamento entre os grupos de cada resíduo, sendo uma roda de discussão com possíveis complementos, considerações e aprimoramento de ideias.
	\end{itemize}
	
	\paragraph{ETAPA 3 (30 minutos)} Dinâmica final dividida no mesmo grupo de pessoas da Dinâmica inicial.
	
	\begin{itemize}
		\item (15 minutos) Discussão interna sobre soluções relacionadas aos problemas elencados na dinâmica inicial e seus possíveis métodos de solução, descrevendo-os em postits separados, intitulados como ‘SOLUÇÕES’ e ‘COMO?’;
		\item (15 minutos) Compartilhamento entre os grupos de cada resíduo, sendo uma roda de discussão com possíveis complementos, considerações e aprimoramento de ideias.
	\end{itemize}
	
	Foram considerados tempos adicionais excedentes, de modo que a oficina totalize um tempo total previsto de 2 horas:
	\begin{itemize}
		\item (10 minutos) Chegada dos participantes e início da Etapa 1;
		\item (10 minutos) Organização da dinâmica e início da Etapa 2;
		\item (10 minutos) Fechamento da Etapa 3 e saída dos participantes.
	\end{itemize}
	
	
	
	\subsection{Resultados}
	As oficinas foram realizadas conforme a Tabela C, com seus respectivos números de participantes e data de participação. A realização no COMTUR foi cancelada pelos integrantes por falta de tempo hábil para realizar a pauta do mês em conjunto com a oficina. No bairro São Benedito houve uma mudança de data devido a um número insuficiente de participantes na primeira oficina, sendo estabelecida uma segunda oficina na semana posterior, ocorrendo divulgação interna nesse período. A lista de presença para cada uma das oficinas realizadas encontra-se no Apêndice B. (ADD NA NOVA VERSÃO DO DOC)
	
	Para aplicação da dinâmica entre os participantes, a divisão se passou em 4 grupos (um para cada resíduo) para oficina com Professores e Bairro Souzas e 2 grupos (um a cada dois resíduos) nas oficinas Centro e São Benedito. As fotos da oficina com os professores encontram-se na Figura 1, no Bairro Centro na Figura X, no Bairro Souzas na Figura X, e por fim, no Bairro São Benedito na Figura C.
	
	\colorbox{green}{figura:Apresentação e dinâmica para os professores da rede pública de Monteiro Lobato}
	
	\colorbox{green}{figura:Apresentação e dinâmica para os moradores do Bairro Centro de Monteiro Lobato}
	
	\colorbox{green}{figura:Apresentação e encerramento com os moradores do Bairro Souzas de Monteiro Lobato}
	
	\colorbox{green}{figura:Apresentação e dinâmica para os moradores do Bairro São Benedito de Monteiro Lobato}
	
	As falas, problemas relatados, sugestão de soluções e comentários foram anotados para cada uma das oficinas em relação a cada tipo de resíduo, assim, foram gerados fluxogramas em função dos resíduos, conglomerando a situação de cada grupo característico de Monteiro Lobato. Os fluxogramas representam os resíduos:
	\begin{itemize}
	\item Figura X: RSU
	\item Figura A: RCC
	\item Figura D: RSS
	\item Figura S: Logística Reversa 
	\end{itemize}
	
	Sendo a coloração dos fluxogramas dada por:
	
	\begin{itemize}
	\item Cor branca: Questão citada em mais de uma oficina
	\item Cor amarela: Questão citada na oficina com os professores
	\item Cor azul: Questão citada na oficina com o Bairro 
	\item Centro Cor verde: Questão citada na oficina com o Bairro Souzas
	\item Cor cinza: Questão citada na oficina com o Bairro São Benedito
	\end{itemize}
	
	Tratando-se do RSU, foi levantado pelos moradores a necessidade de informação a população, dada principalmente pela propaganda, e, em conjunto com a conscientização, fiscalização e incentivo financeiro, promover uma maior motivação da população para com esse tipo de resíduo. Para seu descarte adequado, foi citada a necessidade de adequação, identificação, manutenção e acessibilidade das lixeiras. Por fim, tais ações têm como objetivo uma redução na geração de RSU e uma maior eficiência da coleta de resíduo; resultando em uma não contaminação do meio ambiente e uma possível geração de capital. A totalidade de comentários e sugestões pode ser vista na Figura X.
	
	\colorbox{green}{figura:Fluxograma de RSU}
	
	A partir das quatro oficinas realizadas, as principais percepções em relação ao RCC foram a necessidade de maior estímulo por políticas públicas, principalmente tratando-se de um local adequado de fácil acesso para descarte. Desse modo, possibilita-se o reaproveitamento dos resíduos (por exemplo em seu uso como cascalho nas estradas) e em uma possível geração de renda. A totalidade de comentários e sugestões pode ser vista na Figura X.
	
	\colorbox{green}{figura: Fluxograma de RCC}
	
	Referente ao RSS, encontrou-se quatro principais indicações aos problemas do município, são elas: manter a população informada, presença de fiscalização, lixeiras apropriadas e específicas e um ponto de entrega da prefeitura. Assim feito, espera-se ter um descarte adequado, segurança quanto aos perigos de materiais cortantes e não contaminação do meio ambiente. A totalidade de comentários e sugestões pode ser vista na Figura X.
	
	\colorbox{green}{figura:Fluxograma de RSS (ver resolução)}
	
	Em relação aos resíduos de Logística Reversa, identificou-se como faltantes a presença de informação e conscientização e o interesse dos fabricantes e consumidores. A solução para tais, foi relatada com base em um local próprio para descarte e coleta específica e identificada, acarretando na não contaminação com resíduos perigosos. A totalidade de comentários e sugestões pode ser vista na Figura X.
	
	\colorbox{green}{figura: Fluxograma de Logística Reversa}

	É possível reconhecer uma certa padronização e semelhança entre as sugestões dadas à essas quatro classes de resícuos supracitadas, em que todas convergem ao interesse na gestão de ordens superiores tratando-se de resíduos, local adequado e específico para cada tipo de resíduo e presença de informação e conscientização no município.