\section{Introdução}

Em 02 de dezembro de 2016, o Comitê de Integração da Bacia Hidrográfica do Rio Paraíba do Sul – CEIVAP, instituiu o Plano de Aplicação Plurianual da Bacia Hidrográfica do Rio Paraíba do Sul – PAP, para o período de 2017 a 2020, através da Deliberação CEIVAP nº 237/2016, tendo como base o Plano de Recursos Hídricos da Bacia.

O PAP é o instrumento de planejamento e orientação dos desembolsos a serem executados com recursos da cobrança pelo uso da água, compreendendo os recursos comprometidos, o saldo remanescente até junho de 2016 e aqueles com expectativa de serem arrecadados pela cobrança pelo uso da água de domínio da União e oriundas da transposição do rio Guandu no período de 2017 a 2020.

Com base no PAP, o CEIVAP aprovou a aplicação de recursos financeiros oriundos da cobrança pelo uso da água na bacia para elaboração de Planos Municipais de Gestão Integrada de Resíduos Sólidos - PMGIRS dos municípios integrantes da bacia hidrográfica, por meio do Programa 2.1.3 - Coleta e Disposição de Resíduos Sólidos Urbanos. Os recursos financeiros disponíveis para elaboração de tais planos são provenientes da arrecadação da cobrança pelo uso dos recursos hídricos na Bacia Hidrográfica do Rio Paraíba do Sul.

A Lei Federal nº 12.305 de 2010, que institui a Política Nacional de Resíduos Sólidos, visa a gestão integrada e o gerenciamento adequado dos resíduos sólidos, e introduz o PMGIRS como instrumento de planejamento, com horizonte de 20 anos ou mais, e tem o objetivo principal de promover o diagnóstico da situação atual dos resíduos sólidos no município, bem como prever soluções integradas, tornando-se indispensável para o manejo e a gestão de resíduos sólidos adequados no município.

Além disso, segundo o artigo 18 da Lei nº 12.305/2010, é necessário a elaboração do PMGIRS para que os municípios tenham acesso a recursos da União, ou por ela controlados, bem como incentivos ou financiamentos de entidades federais de crédito ou fomento destinados a serviços relacionados à limpeza urbana e ao manejo de resíduos sólidos.

Este documento corresponde ao primeiro produto do PMGIRS de Monteiro Lobato (SP), e conterá uma análise preliminar da legislação que rege a questão dos resíduos sólidos, bem como a estratégias de mobilização e participação social, conforme o Manual de Referência – Diretrizes para elaboração do Plano Municipal de Gestão Integrada de Resíduos Sólidos (PMGIRS), elaborado pela AGEVAP.
