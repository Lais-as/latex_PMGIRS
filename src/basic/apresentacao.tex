\thispagestyle{headfootimage}

\begin{center}
    {\bfseries\Large\MakeUppercase{Apresentação}}
    \vspace{1.5em}
\end{center}
Este documento compõe o conjunto de relatórios referentes ao \gls{pmgirs} de Monteiro Lobato (SP), e foi elaborado por estudantes e estagiários de engenharia ambiental da UNESP – Campus São José dos Campos, integrantes da Escola de Projetos da \gls{agevap}, com o apoio financeiro do CEIVAP.\vspace{1.5em}

De acordo com o Manual de Referência – Diretrizes para elaboração do \gls{pmgirs}, elaborado pela AGEVAP, os Planos devem ser organizados em Produtos, conforme itens abaixo:

\begin{itemize}
    \item {Produto 1 - Legislação Preliminar;}
    \item {Produto 2 - Caracterização Municipal;}
    \item {Produto 3 - Diagnóstico Municipal Participativo;}
    \item Produto 4 - Prognóstico;
    \item Produto 5 - Versão Preliminar do PMGIRS;
    \item Produto 6 - Versão Final do PMGIRS;
    \item Produto 7 - Manual Operativo do PMGIRS.
\end{itemize}



O Produto 1, objeto deste documento, contempla um breve panorama da situação de resíduos sólidos a níveis federal e estadual, bem como um levantamento e análise da legislação federal, estadual e sua integração com a legislação municipal e decretos regulamentadores, na área de resíduos sólidos, educação ambiental e saneamento básico.\vspace{1.5em}

O Produto 2 apresenta a caracterização municipal de Monteiro Lobato (SP) contendo dados geográficos, como localização, climatologia, geologia, relevo e hidrologia; dados político-administrativos, como distritos, poderes, características urbanas, dispositivos legais de zoneamento urbano e demografia; dados socioeconômicos, como educação, trabalho e renda, saúde, economia, disponibilidade de recursos, além de indicadores sanitários, epidemiológicos e ambientais.\vspace{1.5em}

O Produto 3 consiste em um diagnóstico dos resíduos sólidos, bem como procedimentos operacionais e especificações mínimas a serem adotados em serviços públicos de limpeza urbana e de manejo de resíduos sólidos; indicadores; sistema de cálculo de custos da prestação desses serviços, dentre outras informações. Para elaboração deste produto será realizada oficina com a participação da sociedade, além disso, será aplicado questionário acerca da satisfação dos serviços de limpeza urbana e manejo dos resíduos sólidos. A oficina e o questionário serão descritos em Relatório Técnico, separadamente do produto referido.\vspace{1.5em}

O Produto 4 contempla o prognóstico do município, abarcando principalmente programas, ações de educação ambiental, metas de redução, reutilização, coleta seletiva e reciclagem. Além disso, identifica os passivos ambientais relacionados aos resíduos sólidos e estabelece medidas saneadoras. As ações de emergência e contingência também são contempladas neste produto.\vspace{1.5em}

O Produto 5 corresponde à versão preliminar do PMGIRS abrangendo os dados consolidados das versões anteriores. Compreende o diagnóstico da situação atual dos resíduos sólidos, cenários, metas, diretrizes e estratégias para o cumprimento das metas. O Produto 5 ficará disponível para consulta pública no prazo de 30 dias no site do município e da Agevap.\vspace{1.5em}

O Produto 6 é a versão final do PMGIRS contendo as modificações da versão preliminar apresentada e aprovada através da consulta pública. O mesmo contém o documento de legislação preliminar (Produto 1) consolidado e é discutido em audiência pública.\vspace{1.5em}

O Produto 7 consiste no Manual Operativo do PMGIRS, que deverá discriminar as estratégias e ações necessárias para sua efetiva implementação em curto prazo. Seu conteúdo deverá ser organizado em dois blocos:
\begin{enumerate}[label=\roman*]
	\item Formulação de diretrizes e elaboração de propostas; 
	\item os roteiros para concretização das intervenções selecionadas (modelos tático-operacionais), incluindo sua descrição básica, diagramas e/ou fluxogramas e minutas de normativos legais ou institucionais necessárias para sua consecução.
\end{enumerate}